\documentclass[master=cws,masteroption=ai]{kulemt}
\setup{% Verwijder de "%" op de volgende lijn bij UTF-8 karakterencodering
  inputenc=utf8,
  title={Een gecombineerde calculus voor algebraïsche, scoped en parallelle effecten},
  author={Lander Debreyne},
  promotor={Prof.\,dr.\,ir.\ T. Schrijvers},
  assessor={%TODO
  },
  assistant={Ir. B. van den Berg}}
% Verwijder de "%" op de volgende lijn als je de kaft wil afdrukken
%\setup{coverpageonly}
% Verwijder de "%" op de volgende lijn als je enkel de eerste pagina's wil
% afdrukken en de rest bv. via Word aanmaken.
%\setup{frontpagesonly}

% Kies de fonts voor de gewone tekst, bv. Latin Modern
\setup{font=lm}

% Hier kun je dan nog andere pakketten laden of eigen definities voorzien

% Tenslotte wordt hyperref gebruikt voor pdf bestanden.
% Dit mag verwijderd worden voor de af te drukken versie.
\usepackage[pdfusetitle,colorlinks,plainpages=false]{hyperref}


% Voor wiskundige formules
\usepackage{amsmath}
\usepackage{amsthm}
% pijlen
\usepackage{stmaryrd}
% meer pijlen 
\usepackage{latexsym}
%\includeonly{hfdst-n}
\usepackage{semantic}
\usepackage{soul}
\newcommand{\concat}{\mathbin{{+}\mspace{-8mu}{+}}}
\begin{document}

%TODO
\begin{preface}
  ...
  %Dit is mijn dankwoord om iedereen te danken die mij bezig gehouden heeft.
  %Hierbij dank ik mijn promotor, mijn begeleider en de voltallige jury.
  %Ook mijn familie heeft mij erg gesteund natuurlijk.
\end{preface}

\tableofcontents*

\begin{abstract}
% TODO: schrijf abstract
...
%  In dit \texttt{abstract} wordt een al dan niet uitgebreide
%  samenvatting van het werk gegeven. De bedoeling is wel dat dit tot
%  1~bladzijde beperkt blijft.

\end{abstract}

% Een lijst van figuren en tabellen is optioneel
%\listoffigures
%\listoftables
% Bij een beperkt aantal figuren en tabellen gebruik je liever het volgende:
\listoffiguresandtables
% De lijst van symbolen is eveneens optioneel.
% Deze lijst moet wel manueel aangemaakt worden, bv. als volgt:
% TODO: 
%\chapter{Lijst van afkortingen en symbolen}
%\section*{Afkortingen}
%\begin{flushleft}
%  \renewcommand{\arraystretch}{1.1}
%  \begin{tabularx}{\textwidth}{@{}p{12mm}X@{}}
%    LoG   & Laplacian-of-Gaussian \\
%    MSE   & Mean Square error \\
%    PSNR  & Peak Signal-to-Noise ratio \\
%  \end{tabularx}
%\end{flushleft}
%\section*{Symbolen}
%\begin{flushleft}
%  \renewcommand{\arraystretch}{1.1}
%  \begin{tabularx}{\textwidth}{@{}p{12mm}X@{}}
%    42    & ``The Answer to the Ultimate Question of Life, the Universe,
%            and Everything'' volgens de \cite{h2g2} \\
%    $c$   & Lichtsnelheid \\
%    $E$   & Energie \\
%    $m$   & Massa \\
%    $\pi$ & Het getal pi \\
%  \end{tabularx}
%\end{flushleft}

% Nu begint de eigenlijke tekst
\mainmatter

\chapter{Inleiding} \label{inleiding}
Programmeren omvat het schrijven van instructies die een computer uitvoert om een specifieke taak te volbrengen. Programmeurs schrijven programma's in een programmeertaal, een taal die bestaat uit een set woorden, symbolen en regels. Formele systemen zijn nuttig om de eigenschappen en het gedrag van een programmeertaal te bestuderen. Programmeertaal-calculi bieden een wiskundig kader dat analyse mogelijk maakt om de fundamentele eigenschappen van de taal te begrijpen. Met behulp van abstracte algebra en logische notatie worden de syntaxis, de operationele semantiek en het type-en-effectsysteem van een vereenvoudigde versie van de programmeertaal voorgesteld. De syntaxis beschrijft hoe de woorden en symbolen uit de programmeertaal correct geschreven programma's vormen. De operationele semantiek beschrijft hoe programma's uitgevoerd worden. Het type-en effect-systeem beschrijft hoe de termen in de syntaxis een type krijgen.  

\section{Effecten}
Effecten in programma's zijn interacties met een omgeving buiten de lokale omgeving waarin het programma wordt uitgevoerd. Mogelijke effecten zijn het weergeven van informatie op het scherm, het opslaan van gegevens in een database. In programma's zijn effecten noodzakelijk om het gewenste gedrag te bereiken. Anderzijds kunnen effecten onbedoelde en ongewenste gevolgen hebben. In pure functionele programmeertalen kunnen in principe geen effecten plaatsvinden buiten het uitvoeren van een berekening. Omdat de programmeur geen rekening moet houden met andere effecten kan code geschreven in pure functionele programmeertalen makkelijker te begrijpen en over te redeneren zijn. Deze code verhoogt de productiviteit en verlaagt de kans op fouten. Het puur of vrij van neveneffecten zijn van deze programma's kan limiterend zijn voor de expressiviteit en  het bereiken van het gewenste gedrag. Daarom is het elegant introduceren en correct afhandelen van effecten een belangrijke open vraag in het programmeertaal-onderzoek, in het bijzonder voor functionele programmeertalen. Een goede functionele programmeertaal isoleert, controleert en beheert effecten op een voorspelbare manier. Voor de programmeur maakt een goede functionele programmeertaal het redeneren, uitbreiden, testen en onderhouden van programma's met effecten duidelijk en efficiënt. \emph{Expliciete constructies} voor het redeneren over effecten zijn essentieel om dit doel te bereiken. \newline

\section{Monads en effect handlers}
De meest industrie-relevante aanpak om effecten te modelleren in pure, functionele programmeertalen is de \emph{monad} \cite{Moggi1991}. De bekendste voorbeelden zijn Optional in Java, Result in Rust, en de IO monad en de Maybe monad in Haskell. In Haskell zijn \emph{monad transformers}\cite{Liang1995} populair om modulaire compositie van monads, en bijgevolg effecten, te realiseren. Dit gebeurt door verschillende types monads te combineren tot een enkele, gecombineerde monad. \newline 
Een tweede constructie om effecten te modelleren is het gebruik van \emph{algebraïsche effecten \cite{Pretnar2015} en effect handlers}. Het concept is om de aanroep van het effect, of de syntaxis, te scheiden van de afhandeling van het effect, of de semantiek. Een effect handler kan worden beschouwd als een functie die verantwoordelijk is voor de afhandeling van een effect in een andere functie zodat dit op een voorspelbare en modulaire manier kan gebeuren. 

\section{Programmeertaal-onderzoek en industrie}
WebAssembly \cite{Haas2017} is een goed voorbeeld van hoe principes en vooruitgang uit programmeertaal-onderzoek vertaald kunnen worden naar een industrie-relevante programmeertaal. WebAssembly is van het begin af aan ontworpen met een formele semantiek hetgeen bewijst dat dit een waardevolle aanpak kan zijn. De doelen die de auteurs vooropstellen voor een veilige, snelle, draagbare en compacte taal zijn toepasbaar op het ontwerp van bijna alle talen. Met deze principes in gedachte kan toegewerkt worden naar een calculus voor effect handlers die, mits verder ontwikkeling, kan evolueren naar een industrie-relevante aanpak om in een pure functionele taal met effecten om te gaan. Dit met als doel de programmeur een volwaardig alternatief te bieden voor monads in pure, functionele talen. \newline

Een online bibliografie\footnote{\url{https://github.com/yallop/effects-bibliography}} verzamelt toepassingen en applicaties die gerelateerd zijn aan onderzoek rond  op deze manier programmeren met computationele effecten. Koka \cite{Leijen2017}, Eff \cite{Bauer2015} en Effekt \cite{Brachthauser2020} zijn onderzoeks-programmeertalen ontwikkeld met als doel programmeren met effect handlers. De Haskell-library \emph{fused-effects} \footnote{\url{https://github.com/fused-effects/fused-effects}} biedt functionaliteit om in haskell programma's te maken met effect handlers. Deze library wordt gebruikt door Github in de semantic\footnote{\url{https://github.com/github/semantic}} library. Verder is er de Pyro \cite{bingham2019pyro} library voor flexibel en schaalbaar diep probabilistisch programmeren die volgens de auteurs gebouwd is op Poutine, een library om te programmeren met effect handlers, voor de flexibiliteit en separation of concerns die de aanpak met zich meebrengt.

\section{Gecombineerde calculus}
De meest bestudeerde vorm van effect handlers zijn de handlers die algebraïsche effecten behandelen\cite{Bauer2015}. Algebraïsche effecten zijn effecten die geschreven kunnen worden in de vorm $\textbf{op}\ v\ (y.\ c)$ met $v$ een parameter voor het effect en $(y. \  c)$ voor de continuatie of de rest van het programma na het effect. Een reden waarom deze vorm het meest bestudeerd is, is omdat deze effecten generisch kunnen sequencen en forwarden. \newline
Generische sequencing: 
\begin{equation}
    \inference{}{\textbf{do} \  x \leftarrow \textbf{op} \  l \  v \  (y. \  c_1); \   c_2 \leadsto \textbf{op} \  l \  v \  (y. \   \textbf{do} \  x \leftarrow \  c_1 ; \  c_2)}[E-DoOp]
\end{equation} 
Generische forwarding met h een handler die de operatie met label l niet behandelt:
\begin{equation}
    \inference{(\textbf{op}\:l\:\_\:\_) \notin h}{h \star \textbf{op}\:l\:v\:(y.\:c_{1}) \leadsto \textbf{op}\:l\:v\:(y.\: h \star c_{1})}[E-FwdOp]
\end{equation} 
Calculi van effect handlers zouden ideaal gezien drie eigenschappen moeten bezitten.
\begin{itemize}
    \item Overloading van de effecten: De effecten krijgen een verschillende semantiek door het schrijven en gebruiken van verschillende handlers voor hetzelfde programma.
    \item  Functie compositie: Het modulair combineren van verschillende effecten en effect handlers is mogelijk zonder strikte beperkingen.
    \item Effectinteracties: Effecten interageren met elkaar, waarbij de volgorde van de verschillende handlers een rol speelt en nieuwe semantische mogelijkheden biedt.
\end{itemize}

Het scheiden van operaties en hun behandeling, respectievelijk in de effecten en de handlers, zorgt voor overloading en modulaire compositie. Effectinteracties resulteren uit de compositie van niet-orthogonale effecten. \newline

Algebraïsche effect handlers zijn expressief beperkt in de zin dat ze geen model zijn voor alle computationele effecten. Dit vloeit rechtstreeks voort uit de syntaxis van algebraïsche effecten die als input enkel een parameter en de continuatie van het programma hebben waardoor het semantische domein beperkt is. De expressiviteit van programmeren met algebraïsche effect handlers is beperkt als gevolg hiervan. \newline

Het toevoegen van scoped effecten verhoogt de expressiviteit van een effect handler-gebaseerde programmeertaal. Scoped effecten splitsen het programma in een computatie in scope en een deel dat buiten de scope valt. Het resultaat is dat de handler deze continuatie in scope anders kan behandelen dan de computatie out of scope. Deze syntaxis laat complexere semantiek toe door het mogelijk te maken de computatie in scope een andere betekenis te geven dan de  continuatie. Generisch is de syntaxis te schrijven als $\textbf{sc}\:v\:(y.\:c_1)\:(z.\:c_2)$ met $v$ de parameter, $(y.\:c_1)$ de berekening in scope en $(z.\:c_2)$ de continuatie. De $\lambda_{sc}$ calculus \cite{Bosman2022} beschrijft een calculus voor algebraïsche en scoped effecten. Deze is expressiever dan een calculus voor enkel algebraïsche effecten. De $\lambda_{sc}$ calculus behandelt effecten echter strikt sequentieel en laat daardoor mogelijke performantieverbeteringen naast zich liggen. \newline

Methodiek toevoegen om effecten in parallel te behandelen kan de efficiëntie verbeteren. Parallelle algebraïsche effecten\cite{Xie2021} bieden deze mogelijkheid. De $\lambda^{p}$ calculus maakt parallelle afhandeling van effecten mogelijk door gebruik te maken van een parallelle \emph{for each constructie die effecten parallel behandelt} en levert zo mogelijk performantieverbeteringen voor parallelliseerbare algebraïsche effecten. \newline

\section{Doel}
Het doel van deze masterproef is het afleveren van een calculus met effect handlers die een formeel systeem modelleert met ondersteuning voor algebraïsche en scoped effecten en parallelle afhandeling van effecten. In de literatuur bestaat een leemte voor systemen die deze combinatie expliciet ondersteunen. Deze masterproef wil een ontwerp aanreiken voor een dergelijke calculus. Het resulterende ontwerp steunt op een sterke theoretische basis door een formele syntax, semantiek en type-en effectsysteem af te leveren maar kent ook praktische bruikbaarheid door de implementatie van een functionele interpreter. \newline

De beoogde calculus heeft minimaal volgende meta-eigenschappen: 
\begin{itemize}
    \item \emph{Veilig} voor het programmeren met effecten door het schrijven van overzichtelijke en duidelijke programma's mogelijk te maken door scheiding van syntax en semantiek van effecten door gebruik van de effect handler aanpak.
    \item \emph{Snel} door parallelle afhandeling van effecten mogelijk te maken.
    \item \emph{Compact} door een minimale calculus voor te stellen die aan deze eigenschappen voldoet.
    \item \emph{Draagbaar} door een calculus voor te stellen die wijd implementeerbaar is.
\end{itemize}

 De onderzoeksvraag voor deze masterproef luidt: \newline

\emph{Hoe kan een compacte calculus worden gedefinieerd die sequentiële en parallelle afhandeling van algebraïsche en scoped effecten modelleert met behoud van handler overloading, modulaire compositie en effect interacties?}

\section{Overzicht}
Hoofdstuk \ref{hoofdstuk:achtergrond} geeft meer achtergrondinformatie over de onderwerpen die deze masterproef behandelt. Vervolgens overloopt \Cref{hoofdstuk:startpunt} de literatuur waarop deze thesis voortbouwt. De motivatie voor dit onderwerp en de uitdagingen zijn het onderwerp van \Cref{hoofdstuk:motivatie}. De hoofdstukken die daarop volgen vormen de bijdrage van deze masterproef en stellen de $\lambda_{sc}^{p}$-calculus voor met de syntaxis in \Cref{hoofdstuk:syntaxis}, de operationele semantiek in \Cref{hoofdstuk:semantiek}, het type-en effect-systeem in \Cref{hoofdstuk:typesysteem}, voorbeelden in \Cref{hoofdstuk:voorbeelden} en de metatheorie in \Cref{hoofdstuk:metatheorie}. Vervolgens geeft hoofdstuk \Cref{hoofdstuk:evaluatie} een evaluatie van de masterproef, \Cref{hoofdstuk:gerelateerd} geeft een overzicht van gerelateerd werk. Tot slot bevat \Cref{besluit} de conclusie.

%%% Local Variables: 
%%% mode: latex
%%% TeX-master: "masterproef"
%%% End: 
\chapter{Achtergrond}
\label{hoofdstuk:achtergrond}

\section{Effecten en handlers} \label{sec:achtergrondAlgEff}
Effecten en handlers \cite{Bauer2015} zijn een mechanisme om effecten gestructureerd te behandelen in een programmeertaal. Ze stellen de programmeur in staat om de pure, functionele code in het programma te scheiden van de impure, effectvolle code. Dit gebeurt door effect-operaties te introduceren die effectvolle primitieven zijn. Vervolgens definieert de aanpak specifieke functies, effect handlers, voor het afhandelen van de effect-operaties. Effecten en handlers maken het makkelijker over de logica van de code te redeneren en effecten te coderen en te manipuleren op een modulair componeerbare manier, waardoor meer modulaire en herbruikbare programma's kunnen worden gemaakt. Dit is bijzonder interessant wanneer complexe effecten moeten worden gecontroleerd, zoals in parallelle of gedistribueerde systemen. \newline

Handlers worden gebruikt om de instructies te beschrijven die moeten uitgevoerd worden wanneer een effect optreedt in de controlestroom van het programma. Doordat de semantiek van het effect in de handler zit, maakt dit de scheiding mogelijk tussen het voorkomen van het effect en de afhandeling, waardoor het gemakkelijker wordt te redeneren over de effecten in een programma en deze te beheren. \newline

De belangrijkste eigenschap van programmeren met effect handlers is de hoge mate van modulariteit en diversiteit van semantiek die de programmeur kan bereiken met dezelfde bouwstenen. Dit vloeit voort uit overloading van effecten via verschillende handlers, modulaire compositie van effecten en scheiding van syntax en semantiek bij de afhandeling van effecten. Deze eigenschappen maken dit een veelbelovende aanpak voor het programmeren met effecten. 

\section{Algebraïsche operaties}
Zoals vermeld in Sectie \ref{sec:achtergrondAlgEff} zijn de operaties effectvolle primitieven. Deze primitieven hebben een type-signatuur $A \rightarrowtriangle B$, wat aanduidt dat het effect een term van type A omvormt naar type B. Het $choose:() \rightarrowtriangle Bool$ effect is een voorbeeld van een algebraïsch effect dat een eenheids-waarde neemt als input en op willekeurige of niet-deterministische manier een boolean waarde teruggeeft. De syntaxis hiervoor in deze masterproef is als volgt:

\begin{equation}
    \textbf{op} \: choose \: (\:) \: (y. \: c)
\end{equation}

Hier geeft het sleutelwoord \textbf{op} aan dat het om een algebraïsch effect gaat, gevolgd door het label \textit{choose}, gevolgd door de input parameter $(\:)$, gevolgd door de resumptie $(y. \: c)$. De resumptie is de computatie die de rest van het programma bevat gegeven de resulterende waarde van het effect. Een voorbeeld programma met het \textit{choose} effect is een programma dat ''pasta'' of ''pizza'' teruggeeft afhankelijk van het resultaat:

\begin{equation}
    c_{ND}\:=\:\textbf{op}\:choose\:()\:(x.\:then\:return\:"pasta"\:else\:return\:"pizza"\:)
\end{equation}

Algebraïsche effecten worden gekenmerkt doordat ze aan elkaar gerijgd kunnen worden door gebruik te maken van de algebraïciteits-eigenschap als volgt:

\begin{equation} \label{eq:algFood}
    \begin{split}
        \textbf{do}\:food \leftarrow \textbf{op}\:choose\:()\:(x. \: if \: x \:then\:return\:"pasta"\:else\:return\:"pizza"\:) \: ; \\ \: \textbf{return}\: "We \: are \: eating \: " \concat  food \concat " \: tonight!"  \leadsto \\
        \textbf{op}\:choose\:()\:(x.\: \textbf{do}\:food \leftarrow  if \: x \:then\:return\:"pasta"\:else\:return\:"pizza"\:; \\ \: \textbf{return} "We \: are \: eating \: " \concat  food \concat " \: tonight!" )
    \end{split}
\end{equation}

Algemeen opgesteld, ziet deze eigenschap er zo uit:

\begin{equation}
    \textbf{do}\:x \leftarrow \textbf{op}\:l\:v\:(y.\:c_{1})\:;\:c_{2} \leadsto  \textbf{op}\:l\:v\:(y.\: \textbf{do}\:x \leftarrow c_{1}\:;\:c_{2})
\end{equation}

Het effect van deze eigenschap is enerzijds dat de \textbf{do} clausule wordt doorgeschoven naar de continuatie en anderzijds dat de computatie na de het \textbf{do} statement (de $c_2$ computatie) in de continuatie wordt geduwd.

\section{Algebraïsche effect handlers}
De effect handlers geven semantiek aan de algebraïsche operaties door te definiëren hoe de operaties te interpreteren. Effect handlers hebben drie interessante eigenschappen, overloading van effecten, functie compositie en effectinteracties.

\subsection{Overloading van effecten}
Overloading van effecten is simpel te demonstreren door twee verschillende handlers te definiëren en deze hetzelfde programma te laten behandelen. Het resultaat zal in beide gevallen anders zijn, wat aangeeft dat handlers een interessante invloed hebben op de semantiek van het programma. \newline 
De handler $h_{True}$ selecteert de $true$ tak van de computatie voor het \textit{choose} effect:
\begin{equation}
    \begin{split}
        h_{True} = \textbf{handler} \: & \{ \: \textbf{return}\:x \mapsto \textbf{return}\:x \\
         & , \: \textbf{op}\:choose\:\_\:k \mapsto k \: true\}
    \end{split}
\end{equation}
Deze handler toegepast op het programma uit Eq. \ref{eq:algFood} heeft als resultaat
\begin{equation}
    "We\:are\:eating\:pasta\:tonight!"
\end{equation} 
De handler $h_{Max}$ selecteert beide takken en combineert het resultaat:
\begin{equation}
    \begin{split}
        h_{True} = \textbf{handler} \: & \{ \: \textbf{return}\:x \mapsto \textbf{return}\:[x] \\
        & , \: \textbf{op}\:choose\:\_\:k \mapsto \textbf{do}\:t \leftarrow k \: true\;;\: \textbf{do}\:f \leftarrow k\:false\;\:t \concat f \}
    \end{split}
\end{equation}

Deze handler toegepast op het programma uit Eq. \ref{eq:algFood} zal als resultaat
\begin{equation}
    ["We\:are\:eating\:pasta\:tonight!",\:"We\:are\:eating\:pizza\:tonight!"]
\end{equation} 
hebben. Deze twee resultaten zijn een interessante toepassing van overloading van effecten door verschillende handlers te definiëren voor hetzelfde effect.

\subsection{Functie compositie}
Functie compositie betekent dat het mogelijk moet zijn om programma's te schrijven die verschillende effecten bevatten. 

\subsection{Effect-interacties}
% Voorbeeldgedreven
...

\section{Beperkingen algebraïsche effecten en handlers}
Algebraïsche effecten en handlers zijn beperkt in hun vermogen om bepaalde soorten effecten te modelleren. Ze kunnen effecten modelleren zoals onder andere I/O, het veranderen van de staat van variabelen en niet-determinisme. Algebraïsche effecten kunnen echter geen effecten voorstellen die de controlestroom van een programma veranderen, zoals uitzonderingen (geen catch functionaliteit mogelijk). Deze beperking vloeit voort uit de aard van algebraïsche effecten, die gericht zijn op het modelleren van effecten die abstract kunnen worden voorgesteld als algebraïsche bewerkingen en compositioneel kunnen worden gecombineerd met andere effecten. Controlestroom-veranderende effecten daarentegen vereisen meer verfijnde mechanismen voor hun behandeling, die buiten de mogelijkheden van algebraïsche effecten vallen.

\section{Effecten met scope} \label{hoofdstuk:AchteffScope}
Scoped effecten \cite{Bosman2022}, \cite{Wu2014}, \cite{Yang2022}, \cite{Pirog2018} zijn effecten waarbij het gedrag in scope beperkt is tot een deel van de totale berekening. Dit type effect verdeelt het programma in een computatie die binnen het bereik van het effect valt en een deel dat buiten het bereik valt. De $\lambda_{sc}$ calculus \cite{Bosman2022} beschrijft een calculus voor algebraïsche en scoped effecten. De syntaxis en semantiek van deze calculus wordt beschreven in Sectie \ref{hoofdstuk:startpuntScoped}. De syntaxis van scoped operaties ziet er als volgt uit:
\begin{equation}
    \textbf{sc}\:l\:v\:(y.\:c_{1})\:(z.\:c_{2})
\end{equation}

Hier geeft het sleutelwoord \textbf{sc} aan dat het om een scoped effect gaat, gevolgd door een label $l$, een input parameter $v$, een berekening in scope $(y.\:c_{1})$ en een resumptie $(z.\:c_{2})$. Het verschil met de algebraïsche effecten is de toevoeging van een berekening in scope. Deze toevoeging laat de handler toe om de bereking in scope anders te interpreteren dan de resumptie. Door deze toevoeging kan ook geen generieke forwarding van effecten door verschillende handlers meer gebeuren en moet de handler een expliciete forwarding clausule hebben van de volgende vorm:
\begin{equation}
    \textbf{fwd}\:f\:p\:k \mapsto c_{f}
\end{equation}
Deze clausule wordt meer in detail besproken in Sectie \ref{hoofdstuk:startpuntScoped} en \cite{Bosman2022}.


\section{Parallelle algebraïsche effecten}
Effect handlers behandelen effecten standaard sequentieel. De $\lambda^{p}$ calculus \cite{Xie2021} maakt parallelle afhandeling van algebraïsche effecten mogelijk. Deze calculus wordt behandeld in Sectie \ref{hoofdstuk:startpuntParallel}. De $\lambda^{p}$-calculus maakt gebruikt van een constructie 
\begin{equation}
    \textbf{for}\:x\::\:n.\:e
\end{equation}

Waarbij \textbf{for} het sleutelwoord is, en de expressie $e$ in parallel wordt uitgevoerd voor elke $x$ in $n$. Gelijkaardig aan de expliciete forwarding clausule voor scoped effecten zoals besproken in Sectie \ref{hoofdstuk:AchteffScope}, heeft deze calculus een \textbf{traverse} clausule in de vorm
\begin{equation}
    \textbf{traverse}\:n\:l\:k
\end{equation}
Deze \textbf{traverse} clausule is handler specifiek en verandert de resumptie.

\section{Calculi}
Om een ontwerp voor een programmeertaal met bepaalde eigenschappen of mogelijkheden formeel voor te stellen, kan gebruikt worden gemaakt van een wiskundig kader zoals een calculus. Een calculus stelt een vereenvoudigde versie van een taal voor in abstracte wiskundige notatie en is een elegante manier om programmeertalen te modelleren omdat ook metatheoretische bewijzen van gewenste eigenschappen geleverd kunnen worden. Deze sectie zal een overzicht geven van een simpele calculus, de pure simpel getypeerde lambda-calculus met als doel de verschillende onderdelen van een calculus uiteen te zetten. De calculus is een licht aangepaste versie van de calculus in Hoofdstuk 9 van TAPL \cite{Pierce2002}.

\subsection{Termen, waarden, computaties en operaties}
De kleinste bouwblokken voor programma's in calculi zijn termen. Termen zijn opgesplitst in waarden en computaties waarbij een term bij een van beide hoort. Een waarde is een term die niet kan evalueren of reduceren. Een computatie is in tegenstelling tot een waarde een term die wel kan evalueren of reduceren. Effect handlers introduceren operaties die in deze context duiden op syntax-constructies die de oproep van een effect symboliseren.

\subsection{Syntaxis}
\begin{table}
    \centering
    \begin{tabular}{|r c l r|}
    \hline 
         waarden $v$ & $::=$ & $x$ & variabele \\
         & $|$ & $\lambda\:x:A.\:c$ & abstractie \\
         & & & \\
         computaties $c$ & $::=$ & $v\:v$ & applicatie \\
         & & & \\
        types $A, B$ & $::=$ & $A \rightarrow B$ & type van functies \\
         & & & \\
         \\
         contexten $\Gamma$ & $::=$ & $.$ & Lege context \\
         & $|$ & $\Gamma,\:x:A$ & term variabele binding \\
    \\
    \hline
    \end{tabular}
    \caption{Pure, simpel getypeeerde $\lambda$-calculus syntaxis}
    \label{fig:syntaxisSTLC}
\end{table}

De syntaxis geeft een abstract overzicht van de verschillende bouwstenen waaruit een correct programma kan bestaan. Tabel \ref{fig:syntaxisSTLC} geeft de syntaxis voor de pure, simpel getypeerde $\lambda$-calculus. De syntaxis maakt een onderscheid tussen waarden, computaties, types en contexten en definieert voor elk van deze welke vormen ze kunnen aannemen. Waarden en computaties hebben types. De typering van waarden en computaties is vastgelegd in de type-context. Het verschil tussen waarden en computaties is dat waarden zich in een irreduceerbare, normaalvorm bevinden en computaties kunnen reduceren. De operationele semantiek legt de regels voor reductie vast.

\subsection{Semantiek}

\begin{table}
    \centering
    \begin{tabular}{|l|}
        \hline
        % top
        \\
        % header
         \begin{tabular} {l r}
              \begin{tabular}{|l|}
              \hline
                     $c \leadsto c'$ \\
                \hline
              \end{tabular} & Computatie reductie \\
         \end{tabular} \\
         % rules
          \begin{tabular}{c}
          \\
            $\inference{c_1 \leadsto c_1'}{c_1\:c_2 \leadsto c_1'\:c_2}[E-App1]$ \\
            \\
            $\inference{c \leadsto c'}{v\:c \leadsto v\:c'}[E-App2]$ \\
            \\
            $\inference{}{(\lambda x:A.\:c)\:v \leadsto [x \mapsto v]\:c}[E-AppAbs]$
          \end{tabular} \\
          % bottom
          \\
        \hline
    \end{tabular}
    \caption{Operationele semantiek van pure, simpel getypeerde $\lambda$-calculus}
    \label{fig:semantiekSTLC}
\end{table}

Tabel \ref{fig:semantiekSTLC} toont de operationele semantiek voor de simpel, getypeerde $\lambda$-calculus. Dit zijn semantische regels die gevolgd kunnen worden om berekeningen te reduceren tot normaalvormen. De relatie $c \leadsto c'$ betekent dat de computatie kan reduceren.

\subsection{Type-systeem}

\begin{table}
    \centering
    \begin{tabular}{|l|}
        \hline
        % top
        \\
        % header
         \begin{tabular} {l r}
              \begin{tabular}{|l|}
              \hline
                     $\Gamma \vdash v\::\:A$ \\
                \hline
              \end{tabular} & waarde typering \\
         \end{tabular} 
         \\
         % rules
          \begin{tabular}{c}
          \\
            $\inference{x:A \in \Gamma}{\Gamma \vdash x\::\:A}[T-Var]$ \\
            \\
            $\inference{\Gamma,\:x:A \vdash c\::\:B}{\Gamma \vdash \lambda x:A.\:c\::\:A \rightarrow B}[T-Abs]$ \\
            \end{tabular} \\
            \\
            \begin{tabular} {l r}
              \begin{tabular}{|l|}
              \hline
                     $\Gamma \vdash c\::\:A$ \\
                \hline
              \end{tabular} & computatie typering \\
         \end{tabular} \\
         \\
            \begin{tabular}{c}
            $\inference{\Gamma \vdash v_1 \::\:A \rightarrow B \qquad \Gamma \vdash v_2 \::\:A}{\Gamma \vdash v_1\:v_2\::\:B}[T-App]$
          \end{tabular} \\
          % bottom
          \\
        \hline
    \end{tabular}
    \caption{Typesysteem van pure, simpel getypeerde $\lambda$-calculus}
    \label{tab:semantiekSTLC}
\end{table}

Tabel \ref{tab:semantiekSTLC} toont het typesysteem voor de pure, simpel getypeerde $\lambda$-calculus. Het typesysteem bevat typeregels voor de verschillende mogelijke termen. Dit typesysteem bevat regels voor variabelen (\textbf{T-Var}), abstractie (\textbf{T-Abs}) en applicatie (\textbf{T-App}).

\subsection{Metatheorie}
% TODO: 
Voor deze calculus kan een bewijs van vooruitgang gemaakt worden wat inhoudt dat bewezen kan worden dat elke gesloten, getypeerde term $t$ ofwel een waarde is, of een computatie $c$ en een computatie $c'$ bestaat zodat $c \leadsto c'$ en voor $c'$ hetzelfde geldt. \newline

Voor deze calculus kan eveneens een bewijs van preservatie geleverd worden wat inhoudt dat voor als voor termen $c$ en $c'$ geldt: $\Gamma \vdash c\::\:A$ en $c \leadsto c'$ dan geldt $\Gamma \vdash c' \::\:A$. Dit betekent dat als een term goed getypeerd is, elke geldige reductie van de term ook goed getypeerd is.

\section{Andere computationele effecten}
Deze masterproef focust op de algebraïsche en scoped effecten en de parallelle behandeling van effecten. Dit zijn niet de enige types computationele effecten. Deze sectie bespreekt enkele andere soorten effecten die verder niet behandeld worden in de calculus. 

\subsection{Latente effecten}
Latente effecten \cite{vandenBerg2021} zijn een generieke klasse van effecten waarbij de uitvoer van een computatie uitgesteld wordt. Dit is controle-stroom controlerend mechanisme dat verschilt van de mogelijkheden van algebraïsche en scoped effecten.

\subsection{Asynchrone effecten}
De paper \emph{Asynchronous Effects} \cite{Ahman2020} introduceert een calculus om algebraïsche effecten asynchroon te behandelen. % TODO: langer

%Een hoofdstuk behandelt een samenhangend geheel dat min of meer op zichzelf
%staat. Het is dan ook logisch dat het begint met een inleiding, namelijk
%het gedeelte van de tekst dat je nu aan het lezen bent.

%\section{Eerste onderwerp in dit hoofdstuk}
%De inleidende informatie van dit onderwerp.

%\subsection{Een item}
%De bijbehorende tekst. Denk eraan om de paragrafen lang genoeg te maken en
%de zinnen niet te lang.

%Een paragraaf omvat een gedachtengang en bevat dus steeds een paar zinnen.
%Een paragraaf die maar \'e\'en lijn lang is, is dus uit den boze.

%\section{Tweede onderwerp in dit hoofdstuk}
%Er zijn in een hoofdstuk verschillende onderwerpen. We zullen nu
%veronderstellen dat dit het laatste onderwerp is.

%\subsection{Een item}
%Maak ook geen misbruik van opsommingen. Voor korte opsommingen gebruik je
%geen ``\verb|itemize|'' of ``\texttt{enumerate}'' commando's. Doe dus
%\emph{niet} het volgende:
%\begin{quote}
%  De Eiffeltoren bevat drie verdiepingen:
%  \begin{itemize}
%  \item de eerste;
%  \item de tweede;
%  \item de derde.
%  \end{itemize}
%\end{quote}
%Maar doe:
%\begin{quote}
%  De Eiffeltoren bevat drie verdiepingen: de eerste, de tweede en de derde.
%\end{quote}

%\section{Besluit van dit hoofdstuk}
%Als je in dit hoofdstuk tot belangrijke resultaten of besluiten gekomen
%bent, dan is het ook logisch om het hoofdstuk af te ronden met een
%overzicht ervan. Voor hoofdstukken zoals de inleiding en het
%literatuuroverzicht is dit niet strikt nodig.

%%% Local Variables: 
%%% mode: latex
%%% TeX-master: "masterproef"
%%% End: 

\chapter{Startpunt}\label{hoofdstuk:startpunt}
Deze masterproef heeft als doel een calculus voor te stellen die algebraïsche en scoped effecten en effect handlers als \texttt{first-class citizens} beschouwt met sequentiële en parallelle behandeling voor deze effecten. Het startpunt voor deze calculus is $\lambda_{sc}$\cite{Bosman2022}, een calculus die sequentiële behandeling van scoped en algebraïsche effecten ondersteunt. Voor parallelle behandeling van effecten bestaat de $\lambda^{p}$-calculus\cite{Xie2021}, een calculus die parallelle algebraïsche effecten ondersteunt. Het doel van de thesis is de concepten uit deze laatste calculus te vertalen naar een vorm gelijkaardig aan de eerste calculus en aan deze toe te voegen. Dit hoofdstuk bespreekt beide calculi om de lezer een idee te geven van de bestaande lectuur en wat als eigen contributie nodig is om het doel te bereiken.

\section{\texorpdfstring{$\lambda_{sc}$ :}{} Scoped Effecten} \label{hoofdstuk:startpuntScoped}
Deze sectie stelt de calculus voor algebraïsche en scoped effecten \cite{Bosman2022} die als startpunt dient voor deze masterproef. Deze calculus is gebaseerd op Eff \cite{Bauer2015}. De calculus gebruikt een effect systeem zoals Koka \cite{Leijen2017} om effecten te typeren. De hoofdcontributie van de calculus over Eff is de capaciteit om scoped effecten te modelleren.
\begin{table}
    \centering
    \begin{tabular}{|r c l r|}
    \hline
         & & & \\ 
         waarden $v$ & $::=$ & $() \: \: | \: \: (v_{1}, \: v_{2} ) \: \: | \: \: x \: \: | \: \: \lambda x . \: c \: \: | \: \: h$ & \\
         handlers $h$ & $::=$ & $\textbf{handler} \: \{ \: \: \textbf{return} \: x \mapsto c_{r}$ & return clausule\\
         & & $\qquad \qquad \quad , \: oprs$ & effect  clausules \\
         & & $\qquad \qquad \quad , \: \textbf{fwd} \: f \: p \: k \mapsto c_{f} \: \} $ & forwarding clausule \\
         & & & \\
          effect clausules $oprs$ & $::=$ & . & \\ 
          & $|$ & $\textbf{op} \: l \: x \: k \mapsto c, \: oprs$ & algebraïsche effect clausules \\
           & $|$ & $\textbf{sc} \: l \: x \: p \: k \mapsto c, \: oprs$ & scoped effect clausules \\
        & & & \\
         computaties $c$ & $::=$ & $\textbf{return} \: v$ & return waarde \\
          & $|$ & $\textbf{op} \: l \: v \: (y. \: c)$ & algebraïsch effect \\
          & $|$ & $\textbf{sc} \: l \: v \: (y. \: c_{1}) \: (z. \: c_{2})$ & scoped effect \\
          & $|$ & $v \star c$ & behandeling \\
          & $|$ & $\textbf{do} \: x \leftarrow c_{1}\:; \: c_{2}$ & do clausule \\
          & $|$ & $v_{1} \: v_{2}$ & applicatie \\
          & $|$ & $\textbf{let} \: x = v \: \textbf{in} \: c$ & let \\
         & & & \\
         waarde types $A, \: B, \: M$ & $::=$ & $() \: \: | \: \: (A, \:B) \: \: | \: \: A \rightarrow \underline{C} \: \: | \: \: \underline{C} \Rightarrow \underline{D}$ & \\
         & $|$ & $\alpha$ & type variabele \\
         & $|$ & $\lambda \: \alpha . \: A$ & type operator abstractie \\
         & $|$ & $M \: A$ & type applicatie \\
         type schemas $\sigma$ & $::=$ & $A \: \: | \: \: \forall \: \mu . \: \sigma \: \: | \: \: \forall \: \alpha. \: \sigma $ & \\
         computatie types $\underline{C}, \: \underline{D}$ & $::=$ & $A ! \langle E \rangle $ & \\
         effect type rows $E, \: F$ & $::=$ & $. \: \: | \: \: \mu \: \: | \: \: l; \: E $ & \\
         & & & \\
         kinds $K$ & $::=$ & $* \: \: | \: \: K \rightarrow K$ & \\
         & & & \\
         signatuur contexten $\Sigma$ & $::=$ & $. \: \: | \: \: \Sigma , \: l \: : \: A \rightarrowtriangle B$ & \\
         type contexten $\Gamma$ & $::=$ & $. \: \: | \:\: \Gamma, \: x \: : \: A \: \: | \: \: \Gamma , \: \mu \: \: | \: \: \Gamma, \: \alpha $ & \\
         & & & \\
    \hline
    \end{tabular}
    \caption{Syntaxis voor $\lambda_{sc}$}
    \label{fig:syntaxisScoped}
\end{table}

\subsection{Lopend voorbeeld}
Om de syntaxis en de operationele semantiek van deze calculus te illustreren gebruikt deze sectie een lopend voorbeeld. 
Het lopend voorbeeld, voorgesteld in Eq. \ref{eq:runEx}, is gebaseerd op het \emph{Forwarding} voorbeeld uit \cite{Bosman2022}. In dit voorbeeld komen alle relevante reductieregels aan bod. Het voorbeeld is een programma dat het \textbf{once} effect, dat van een computatie met het \textbf{choose} (non-determinisme) effect enkel het eerste resultaat teruggeeft, combineert met het \textbf{inc} effect, dat een teller bijhoudt. Het \textbf{inc} effect transformeert een computatie in een functie die een staat doorgeeft. Dit voorbeeld is gelijkaardig aan het voorbeeld beschreven in Sectie \ref{subsec:effInter}. Om dit voorbeeld te bespreken is het nodig eerst de syntaxis van de calculus te introduceren.

\begin{equation} \label{eq:runEx}
    h_{once} \star (run_{inc} \: 0 \star c_{fwd})
\end{equation}



\subsection{Syntaxis}
Tabel \ref{fig:syntaxisScoped} beeldt de syntaxis voor de $\lambda_{sc}$-calculus af. De syntaxis onderscheidt visueel de termen in de bovenste helft en de types in de onderste helft van de tabel. De termen zijn gesplitst in waarden $v$ en computaties $c$ waarbij waarden verschillen van computaties in dat de semantiek in Sectie \ref{sec:OpSemScop} enkel reducties op computaties definieert. De volgende paragrafen bespreken de termen in de syntaxis die tot de waarden, de computaties en de types behoren.

\subsubsection{Waarden}
We zien dat de syntaxis vijf soorten waarden heeft. Deze overlopen we kort.
Een waarde is ofwel de \emph{eenheids-waarde} $()$ (unit-waarde), een \emph{paar} van twee waarden $(v_{1}, v_{2})$, een \emph{variabele} $x$, een \emph{lambda functie} $\lambda x.\:c$ of een \emph{handler} $h$. De eerste vier soorten waarden zijn weinig interessant. \newline
De belangrijkste soort waarden in een calculus voor effect handlers zijn de \emph{handlers}. Een handler bestaat uit een \emph{return} clausule, geen tot meerdere \emph{effect} clausules en een \emph{forwarding} clausule. In vergelijking met Eff merken we de toevoeging van de forwarding clausule op die specifiek is voor de scoped effecten. \newline 
De forwarding clausule $\textbf{fwd}\:f\:p\:k \mapsto c_{f}$, die elke handler moet implementeren, zorgt voor expliciete forwarding wanneer de te behandelen scoped effect clausule niet geïmplementeerd wordt door de handler. De computatie $c_{f}$ specifieert hoe de onbekende scoped effect clausule moet aangepast worden en kan hiervoor gebruik maken van $f$, dat het onbekende scoped effect bevat, de computatie in scope $p$ en de resumptie $k$. \newline
De return clausule $\textbf{return}\:x \mapsto c_{r}$ duidt aan dat de behandeling van een \textbf{return} computatie met resultaat x door de handler deze computatie vervangt door de handler-bepaalde computatie $c_{r}$. \newline
Zoals aangehaald bestaat een handler uit geen tot meerdere effect clausules. In het geval van deze kunnen deze effect clausule algebraïsche operaties of scoped operaties zijn, waarbij de scoped operaties aan uitbreiding vormen aan de Eff calculus. Een \emph{algebraïsche} effect clausule ($\textbf{op}\:l\:x\:k \mapsto c$) wordt gekenmerkt door het sleutelwoord \textbf{op} gevolgd door een label $l$, een input parameter $x$ en een resumptie k. $\mapsto c$ duidt aan dat een handler deze clausule verwerkt door de computatie $c$ toe te passen. Een \emph{scoped} effect clausule ($\textbf{sc}\:l\:x\:p\:k \mapsto c$) is gelijkaardig aan een algebraïsche effect clausule met een verschillend sleutelwoord \textbf{sc} en bijkomende een bewerking in scope $p$.

\subsubsection{Computaties}
Deze paragraaf bespreekt de zeven computaties die syntactisch mogelijk zijn in de calculus. De effect operaties en behandeling van computaties zijn hierbij het interessantst voor de effect handler aanpak. \newline
De twee operaties die in de calculus gedefinieerd zijn, zijn de operatie voor het algebraïsch effect en die voor het scoped effect. Een \emph{algebraïsche effect operatie} ($\textbf{op}\:l\:v\:(y.\:c)$) bestaat uit het sleutelwoord \textbf{op}, een label $l$, een input parameter waarde $v$ en een resumptie in de vorm $(y.\:c)$. Een \emph{scoped effect operatie} ($\textbf{sc}\:l\:v\:(y.\:c_{1})\:(z.\:c_{2})$) bestaat uit het sleutelwoord \textbf{sc}, een label $l$, een input parameter waarde $v$, een scoped berekening $(y.\:c_{1})$ en een resumptie $(z.\:c_{2})$. In beide soorten effecten is de resumptie de computatie die gegeven de resulterende waarde van het effect, de rest van het programma bevat. \newline  
De behandeling ($v \star c$) drukt de behandeling van een computatie $c$ door een waarde $v$ door middel van de $\star$-operator. Deze waarde is in de praktijk altijd een handler, anders is het programma niet goed getypeerd. \newline 
Computaties kunnen een waarde teruggeven en in normaalvorm belanden door middel van de \emph{return clausule} ($\textbf{return}\:v$). \newline  
Computaties worden aan elkaar gerijgd doormiddel van do clausules in de vorm $\textbf{do}\:x \leftarrow c_{1};\:c_{2}$ waarbij het resultaat van $c_{1}$ in $c_{2}$ beschikbaar is als $x$. \newline 
De syntax van applicatie in de calculus is door $v_{1}\:v_{2}$. \newline
Let-polymorfisme is beschikbaar in de calculus als $\textbf{let}\ 
 x \  = \  v \  \textbf{in} \  c$. Dit is een uitbreiding op Eff.

\subsubsection{Lopend voorbeeld}
Het lopend voorbeeld (Eq. \ref{eq:runEx}) introduceert de handlers $h_{once}$ en $h_{inc}$. $c_{fwd}$ is de te behandelen computatie en bestaat uit een scoped effect operatie met in de continuatie een algebraïsch effect operatie. De volgende vergelijkingen stellen deze computaties, handlers en operaties voor in de syntaxis van de calculus.

\begin{equation}
    c_{fwd} \: = \: \textbf{sc}\:once\:()\:(\_.\:c_{inc})\:(x.\:\textbf{op}\:inc\:()\:(y.\:\textbf{return}\:(x+y)))
\end{equation}

\begin{equation}
    run_{inc} \:s\:c \equiv \textbf{do}\:c' \leftarrow h_{inc} \star c;\:c'\:s
\end{equation}

\begin{equation}
    \begin{split}
        h_{inc} = \textbf{handler}\: & \{\:\textbf{return}\:x \mapsto \textbf{return}\:(\lambda s.\: \textbf{return}\:(x,\:s)) \\
        & ,\:\textbf{op}\:inc\:\_\:k \mapsto \textbf{return}\:(\lambda s. \: k\:s(s+1)) \\
        & ,\: \textbf{fwd}\:f\:p\:k \mapsto \textbf{return}\:(\lambda s.\:f(\lambda y.\:p\:y\:s,\:\lambda (z,\:s').\: k\:z\:s'))\}
    \end{split}
\end{equation}

\begin{equation}
    \begin{split}
        h_{once} = \textbf{handler}\: & \{\:\textbf{return}\:x \mapsto \textbf{return}\:[x] \\
        & ,\:\textbf{op}\:choose\:\_\:k \mapsto \textbf{do}\:xs \leftarrow k\:true;\:\textbf{do}\:ys \leftarrow k\:false;\:xs \concat ys \\
        & ,\:\textbf{sc}\:once\:\_\:p\:k \mapsto \textbf{do}\:ts \leftarrow p\:();\:\textbf{do}\:t \leftarrow head \:ts;\:k\:t \\
        & ,\:\textbf{fwd}\:f\:p\:k \mapsto f\:(p,\:(\lambda z.\: concatMap \:z\:k))\}
    \end{split}
\end{equation}
De concatMap functie wordt in dit voorbeeld niet besproken maar wel in de paper over de $\lambda_{sc}$-calculus\cite{Bosman2022}.

\subsubsection{Types}
%TODO: uitbreidingen tov Eff
Analoog aan de termen zijn ook de types opgesplitst in waarde types $A,\:B,\:M$ en computatie types $\underline{C},\:\underline{D}$. \newline
De calculus bevat enkele waarde types die ook terug te vinden zijn in Eff, namelijk een \emph{eenheids-type} $()$, een \emph{paar-type} $(A,\:B)$, een \emph{functie type} $A \rightarrow \underline{C}$ en een \emph{handler type} $\underline{C} \Rightarrow \underline{D}$. Verder is de syntaxis uitgebreid tegenover Eff door toevoeging van een \emph{type variabele} $\alpha$, een \emph{type operator abstractie} $\lambda \: \alpha. \: A$ een een \emph{type applicatie} $M\:A$. \newline 
Een computatie type heeft een enkele vorm ($A!\langle E \rangle$) en bestaat uit een \emph{waarde type} $A$, het \emph{type van de resulterende waarde} van de computatie, en een \emph{effect type} $E$. Het effect type is het type van de effecten die tijdens de computatie kunnen opgeroepen worden. \newline 
Een effect type of \emph{effect row} bestaat uit een mogelijke lege collectie atomische labels $l$, mogelijk met een row-variabele $\mu$ als laatste element. \newline
Deze calculus ondersteunt verder \emph{type schemas} $\sigma$, waarbij applicatie van type variabelen op type schemas een uitbreiding vormt en \emph{kinds} $K$ die ook niet beschikbaar zijn in Eff. De calculus houdt een gespecialiseerde \emph{signatuur context} $\Sigma$ bij om via hun labels de signatuur van gebruikte effecten te typeren. De calculus heeft ook een \emph{type context} die de types van variabelen bijhoudt.

\subsection{Operationele semantiek}
\label{sec:OpSemScop}

\begin{table}
    \centering
    \begin{tabular}{|l|}
        \hline
        % top
        \\
        % header
         \begin{tabular} {l r}
              \begin{tabular}{|l|}
              \hline
                     $c \leadsto c'$ \\
                \hline
              \end{tabular} & Reductie van computaties \\
         \end{tabular} \\
         % rules
          \begin{tabular}{c}
            $\inference{}{(\lambda x.\:c)\:v \leadsto c\:[\:v\:/\:x\:]}[E-AppAbs] \qquad \inference{}{\textbf{let}\:x\: = \: v \: \textbf{in} \:c \leadsto c\:[\:v\:/\:x\:]}[E-Let]$ \\ 
            \\
            $\inference{c_{1} \leadsto c_{1}'}{\textbf{do}\:x \leftarrow c_{1}\:;\:c_{2} \leadsto \textbf{do}\:x \leadsto c_{1}'\:;\:c_{2}}[E-Do] \qquad \inference{}{\textbf{do}\:x \leftarrow \textbf{return}\:v\:;\:c_{2} \leadsto c_{2}\:[\:v\:/\:x\:]}[E-DoRet]$ \\
            \\
            $\inference{}{\textbf{do}\:x \leftarrow \textbf{op}\:l\:v\:(y.\:c_{1})\:;\:c_{2} \leadsto \textbf{op}\:l\:v\:(y.\: \textbf{do} \: x \leftarrow c_{1}\:;\:c_{2})}[E-DoOp]$ \\
            \\
            $\inference{}{\textbf{do}\:x \leftarrow \textbf{sc} \:l\:v\:(y.\:c_{1})\:(z.\:c_{2}) \: ;\: c_{3} \leadsto \textbf{sc} \:l\:v\:(y.\:c_{1})\:(z.\: \textbf{do} \: x \leftarrow c_{2} \: ;\: c_{3})}[E-DoSc]$ \\
            \\
            $\inference{c \leadsto c'}{h \star c \leadsto h \star c'}[E-Hand] \qquad \inference{(\textbf{return}\:x \mapsto c_{r}) \in h}{h \star \textbf{return} \: v \leadsto c_{r} \:[\:v\:/\:x\:]}[E-HandRet]$ \\
            \\
            $\inference{(\textbf{op}\:l\:x\:k \mapsto c) \in h}{h \star \textbf{op}\:l\:v\:(y.\:c_{1}) \leadsto c\:[\:v\:/\:x,\:(\lambda y.\:h \star c_{1}) \: / \: k]}[E-HandOp]$ \\
            \\
            $\inference{(\textbf{op}\:l\:\_\:\_) \notin h}{h \star \textbf{op}\:l\:v\:(y.\:c_{1}) \leadsto \textbf{op}\:l\:v\:(y.\: h \star c_{1})}[E-FwdOp]$ \\
            \\
            $\inference{(\textbf{sc}\:l\:x\:p\:k \mapsto c) \in h}{h \star \textbf{sc}\:l\:v\:(y.\:c_{1})\:(z.\:c_{2}) \leadsto c\:[\:v\:/\:x,\:(\lambda \: y. \: h \star\:c_{1}) \: / \: p, (\lambda z. \: h \star c_{2}) \:/\:k\:]}[E-HandSc]$ \\
            \\
            $\inference{(\textbf{sc}\:l\:\_\:\_\:\_) \notin h \\ (\textbf{fwd}\:f\:p\:k \mapsto c_{f}) \in h \qquad g\:=\:\lambda(p',\:k')\:.\:\textbf{sc}\:l\:v\:(y.\:p'\:y)\:(z.\:k'\:z)}{h \star \textbf{sc}\:l\:v\:(y.\:c_{1})\:(z.\:c_{2}) \leadsto c_{f}\:[\:(\lambda\:y.\:h \star c_{1})\:/\:p,\:(\lambda z.\: h \star c_{2})\:/\:k,\:g\:/\:f]}[E-FwdSc]$\\
          \end{tabular} \\
          % bottom
          \\
        \hline
    \end{tabular}
    \caption{Operationele semantiek van $\lambda_{sc}$}
    \label{fig:semantiekScoped}
\end{table}
Tabel \ref{fig:semantiekScoped} toont de kleine-staps operationele semantiek van de $\lambda_{sc}$-calculus. De relatie $c \leadsto c'$ duidt aan dat $c$ naar $c'$ stapt of reduceert. Dat dit een computatie reductie relatie is, betekent dat de calculus enkel termen die computaties zijn, kan reduceren. \newline
De regels \textbf{E-AppAbs} en \textbf{E-Let} behandelen functie applicatie en let-binding. De andere regels zijn op te delen in twee domeinen: computaties aan elkaar rijgen en computaties behandelen met handlers.
\subsubsection{Computaties aan elkaar rijgen}
Om computaties aan elkaar te rijgen van de vorm $\textbf{do}\:x \leftarrow c_{1}\:;\:c_{2}$ is een onderscheid te maken tussen het geval waarbij \emph{$c_{1}$ een stap kan zetten} naar $c_{1}'$ (\textbf{E-Do}) of waarbij \emph{$c_{1}$ in normale vorm is} (\textbf{return}, \textbf{op \textellipsis}, of \textbf{sc \textellipsis}). \newline 
In het laatste geval, hangt de regel af van de normale vorm die zich presenteert in $c_{1}$. \newline 
In het geval dat de computatie \emph{een resultaat is} met een waarde, $\textbf{do}\:x \leftarrow \textbf{return}\:v\:;\:c_{2}$, wordt $x$ vervangen door $v$ in $c_{2}$ of $\leadsto c_{2}\:[\:v\:/\:x\:]$ (\textbf{E-DoRet}). \newline
Is de computatie \emph{een algebraïsche effect} ($\textbf{do} \  x \leftarrow \textbf{op} \  l \  v \  (y. \  c_{1}) \  ; \  c_{2}$) dan kan de computatie herschreven worden met behulp van de algebraïciteits-eigenschap $\leadsto \textbf{op} \  l \  v \  (y. \   \textbf{do} \   x \leftarrow c_{1} \  ; \  c_{2})$ (\textbf{E-DoOp}). \newline
Bij \emph{een scoped effect} ($\textbf{do}\:x \leftarrow \textbf{sc} \:l\:v\:(y.\:c_{1})\:(z.\:c_{2}) \: ;\: c_{3}$) wordt een generalisatie van de algebraïciteits-eigenschap gebruikt $\leadsto \textbf{sc} \:l\:v\:(y.\:c_{1})\:(z.\: \textbf{do} \: x \leftarrow c_{2} \: ;\: c_{3})$ (\textbf{E-DoSc}).

\subsubsection{Computaties behandelen met handlers}
Om computaties te behandelen met handlers, $h \star c$, is er opnieuw onderscheid te maken tussen het geval waar \emph{$c$ kan reduceren} ($c \leadsto c'$) en waar \emph{$c$ in normale vorm is} (\textbf{return} v, \textbf{op \textellipsis}, \textbf{sc \textellipsis}). \newline 
In het eerste geval wordt de mogelijke stap genomen ($h \star c \leadsto h \star c'$) (\textbf{E-Hand}). In het andere geval wordt de clausule die geldt uit de handler toegepast (\textbf{E-HandRet}, \textbf{E-HandOp}, \textbf{E-FwdOp}, \textbf{E-HandSc}, \textbf{E-FwdSc}).

\subsubsection{Lopend voorbeeld}
% TODO: beter?
Dit deel van het lopend voorbeeld illustreert de toepassing van de inferentie regels uit de operationele semantiek om de reductie van het voorbeeld tot de uiteindelijk resulterende waarde te doen. De uitgelichte stappen gebruiken enkele van de regels die uniek zijn aan deze calculi of meer algemeen calculi van effect handlers. De hele reductie is hier niet gegeven.\newline
Door $run_{inc}$ te substitueren door de definitie herschrijft het voorbeeld zich als:
\begin{equation} 
    h_{once} \star (run_{inc}\:0 \star c_{fwd}) \equiv h_{once} \star (\textbf{do}\:p' \leftarrow h_{inc} \star c_{fwd};\: p'\:0)
\end{equation}
Door vervolgens \textbf{E-FwdSc} te gebruiken, in combinatie met \textbf{E-Hand} en \textbf{E-Do}, reduceert het voorbeeld met de \textbf{fwd} clausule in de \emph{$h_{once}$} handler.
\begin{equation}
    \begin{split}
        \leadsto & \{-\text{E-Hand and E-Do and E-FwdSc}-\} \\
        & h_{once} \star (\textbf{do}\:p' \leftarrow \textbf{return}\:(\lambda c.\:(\lambda (p,\:k).\: \textbf{sc}\:once\:()\:(y.\:p\:y)\:(z.\:k\:z)) \\
        &  \qquad \qquad \qquad \qquad \qquad \qquad (\lambda y. \: (\lambda \_ .\: h_{inc} \star c_{inc})\:y\:c, \\
        & \qquad \qquad \qquad \qquad \qquad \qquad \lambda (z,\:c').\:(\lambda x.\: h_{inc} \star \textbf{op}\:inc\:()\:(y.\:\textbf{return}\:(x+y)))\:z\:c')); \\
        & \qquad \qquad \qquad \qquad \qquad \qquad p'\:0)
    \end{split}
\end{equation}
Verder in de reductie behandelt de $h_{once}$ handler scoped operatie
\begin{equation}
    \begin{split}
        \leadsto* \: h_{once} \star (\textbf{sc}\:once\:()\: & (y.\:(\lambda y.\:(\lambda \_.\:h_{inc} \star c_{inc})\:y\:0)\:y) \\
        & (z.\:(\lambda (z,\:c').\:(\lambda x.\:h_{inc} \star \textbf{op}\:inc\:()\:(y.\:\textbf{return}\:(x+y)))\:z\:c')\:z)) \\
        \leadsto \{- \text{E-HandSc} -\} & \\
        \quad \textbf{do}\:ts \leftarrow (\lambda y.\: h_{once} & \star (\lambda y.\: (\lambda \_.\:h_{inc} \star c_{inc})\:y\:0)\:y)\:(); \\
        \quad \textbf{do}\:t \leftarrow head\:ts; & \\
        \quad (\lambda z.\:h_{once} \star ( \lambda (z, & \:c'). \:(\lambda x.\:h_{inc} \star \textbf{op}\:inc\:()\:(y.\:\textbf{return}\:(x+y)))\:z\:c')\:z)\:t \\
        \leadsto* \: \textbf{return}\:[(1,\:2)]
    \end{split}
\end{equation}

\subsection{Type- en effect-systeem}
De $\lambda_{sc}$-calculus heeft een apart type-systeem voor de termen en de effecten. Het effect-systeem maakt gebruikt van \emph{effect rows} volgens de stijl van Koka\cite{Leijen2017}. De volgende paragrafen behandelen het type-systeem, met in Tabel \ref{fig:typeWaarde} de waarde typering, in Tabel \ref{fig:typeComp} de computatie typering en in Tabel \ref{fig:typeHand} de handler typering.
\subsubsection{Waarde typering}
\begin{table}
    \centering
    \begin{tabular}{|l|}
        \hline
        % top
        \\
        % header
         \begin{tabular} {l r}
              \begin{tabular}{|l|}
              \hline
                     $\Gamma \vdash v\::\:\sigma$ \\
                \hline
              \end{tabular} & Waarde typering \\
         \end{tabular} \\
         % rules
          \begin{tabular}{c}
          \\
            $\inference{(x\::\:\sigma) \in \Gamma}{\Gamma \vdash x\::\:\sigma}[T-Var] \qquad \inference{}{\Gamma \vdash (\:) \::\:(\:)}[T-Unit]$ \\ 
            \\
            $\inference{\Gamma \vdash v_1\::\:A \qquad \Gamma \vdash v_2 \::\:B}{\Gamma \vdash (v_1,\:v_2)\::\:(A,\:B)}[T-Pair]$\\
            \\
            $\inference{\Gamma,\:x\::\:A \vdash c\::\:\underline{C}}{\Gamma \vdash \lambda \: x.\:c\::\:A \rightarrow \underline{C}}[T-Abs] \qquad \inference{\Gamma \vdash v\::\:A \qquad A \equiv B}{\Gamma \vdash v\::\:B}[T-EqV]$\\
            \\
            $\inference{\Gamma \vdash A \::\: * \qquad \Gamma \vdash v\::\: \forall \: \alpha.\:\sigma}{\Gamma \vdash v\::\:\sigma\:[\:A\:/\:\alpha\:]}[T-Inst] \qquad \inference{\Gamma,\:\alpha \vdash v\::\:\sigma \qquad \alpha \notin \Gamma}{\Gamma \vdash v\::\:\forall\:\mu.\:\sigma}[T-Gen]$\\
            \\
            $\inference{\Gamma \vdash v\::\:\forall\:\mu.\:\sigma}{\Gamma \vdash b \::\:\sigma\:[\:E\:/\:\mu\:]}[T-InstEff] \qquad \inference{\Gamma,\:\mu \vdash v\::\:\sigma \qquad \mu \notin \Gamma}{\Gamma \vdash v\::\:\forall \mu.\:\sigma}[T-GenEff]$\\
            \\
          \end{tabular} \\
          % bottom
          \\
        \hline
    \end{tabular}
    \caption{Waarde typering van $\lambda_{sc}$}
    \label{fig:typeWaarde}
\end{table}

Tabel \ref{fig:typeWaarde} toont de waarde typeringsregels voor $\lambda_{sc}$. \newline 
\textbf{T-Var}, \textbf{T-Unit} en \textbf{T-Pair} behandelen de standaard \emph{typering van respectievelijk variabelen, de eenheidswaarde en paren}. \newline
\textbf{T-Abs} is ook relatief standaard en bespreekt de \emph{typering van een abstractie}. \newline
\textbf{T-EqV} beschrijft een \emph{equivalentie-relatie voor waarden}. De volledige type equivalentie is hier achterwege gelaten maar te vinden in de appendix van de originele paper \cite{Bosman2022}. \newline 
De resterende vier type-regels zijn minder standaard en nieuw voor de calculus tegenover Eff. Deze regels bespreken de \emph{instantie en abstractie over respectievelijk type variabelen en row variabelen} (voor effecten).
\textbf{T-Inst} en \textbf{T-InstEff} behandelen daarbij de instantie van respectievelijk type- en row-variabelen. \textbf{T-Gen} en \textbf{T-GenEff} behandelen dan abstractie over respectievelijk type- en row-variabelen.

\subsubsection{Computatie typering}
\begin{table}
    \centering
    \begin{tabular}{|l|}
        \hline
        % top
        \\
        % header
         \begin{tabular} {l r}
              \begin{tabular}{|l|}
              \hline
                     $\Gamma \vdash c\::\:\underline{C}$ \\
                \hline
              \end{tabular} & Computatie typering \\
         \end{tabular} \\
         % rules
          \begin{tabular}{c}
          \\
            $\inference{\Gamma \vdash v_1 \::\:A \rightarrow \underline{C} \qquad \Gamma \vdash v_2 \::\:A}{\Gamma \vdash v_1\:v_2\::\:\underline{C}}[T-App]$ \\
            \\
            $\inference{\Gamma \vdash c_1\::\:A!\langle E \rangle \qquad \Gamma,\:x\::\:A \vdash c_2 \::\:B! \langle E \rangle}{\Gamma \vdash \textbf{do}\:x \leftarrow c_1;c_2\::\:B!\langle E \rangle}[T-Do]$\\
            \\
            $\inference{\Gamma \vdash c\::\:\underline{C} \qquad \underline{C} \equiv \underline{D}}{\Gamma \vdash c\::\:\underline{D}}[T-EqC]$ \\
            \\
            $\inference{\Gamma \vdash v\::\:\sigma \qquad \Gamma,\:x\::\:\sigma \vdash c\::\:\underline{C}}{\Gamma \vdash \textbf{let}\:x = v\:\textbf{in}\:c\::\:\underline{C}}[T-Let]$\\
            \\
            $\inference{\Gamma \vdash v\::\:A}{\Gamma \vdash \textbf{return}\:v\::\:A!\langle E \rangle}[T-Ret]$\\
            \\
            $\inference{\Gamma \vdash v\::\:\forall \:\alpha.\:\alpha!\langle E \rangle \Rightarrow M\:\alpha!\langle F \rangle \qquad \Gamma \vdash c\::\:A!\langle E \rangle}{\Gamma \vdash v \star c\::\:M\:A!\langle F \rangle}[T-Hand]$\\
            \\
            $\inference{(l\::\:A_l \rightarrowtriangle B_l) \in \Sigma \qquad \Gamma \vdash v\::\:A_l \qquad \Gamma,\:y\::\:B_l \vdash c\::\:A!\langle l;E \rangle}{\Gamma \vdash \textbf{op}\:l\:v\:(y.\:c)\::\:A!\langle l;E \rangle}[T-Op]$\\
            \\
            $\inference{(l\::\:A_l \rightarrowtriangle B_l) \in \Sigma \qquad \Gamma \vdash v\::\:A_l \\ \Gamma,\:y\::\:B_l\vdash c_1 \::\: B!\langle l;E \rangle \qquad \Gamma,\:z\::\:B \vdash c_2 \::\:A!\langle l;E \rangle}{\Gamma \vdash \textbf{sc}\:l\:v\:(y.\:c_1)\:(z.\:c_2)\::\:A!\langle l;E \rangle}[T-Sc]$\\
            \\
          \end{tabular} \\
          % bottom
          \\
        \hline
    \end{tabular}
    \caption{Computatie typering van $\lambda_{sc}$}
    \label{fig:typeComp}
\end{table}
Tabel \ref{fig:typeComp} toont de computatie typeringsregels voor $\lambda_{sc}$. \newline
\textbf{T-App, T-Do, T-Ret, T-Let} zijn relatief standaard regels voor respectievelijk \emph{applicatie}, een \emph{\textbf{do} statement}, een \emph{\textbf{return} statement} en \emph{\textbf{let} polymorfisme}. \newline 
Voor de \textbf{T-EqC} regel die \emph{type-equivalentie voor computaties} behandelt geldt dezelfde opmerking als gemaakt voor typering van waarden. \newline
\textbf{T-Hand} typeert de \emph{applicatie van een handler}. De resterende typering voor handlers en bijhorende clausules is gegeven in Tabel \ref{fig:typeHand}. Verschillend aan Eff, typeert $\lambda_{sc}$ hier een polymorfe handler. \newline
\textbf{T-Op} typeert een \emph{algebraïsch effect}. Merk op dat het subscript $l$ hier staat voor de signatuur bekomen door het zoeken naar het label $l$ in de context. \newline
\textbf{T-Sc} typeert een \emph{scoped effect}. Een soortgelijke opmerking als voor algebraïsche effecten geldt hier met het verschil dat de opgevraagde signatuur de signatuur voor de \emph{scoped computatie} is, waardoor het type van de resulterende continuatie door de signatuur onbeschreven is maar de effect rows moeten wel overeenkomen. 

\subsubsection{Handler typering}
\begin{table}
    \centering
    \begin{tabular}{|l|}
        \hline
        % top
        \\
        % header
         \begin{tabular} {l l l r}
              \begin{tabular}{|l|}
              \hline
                     $\Gamma \vdash oprs\::\:\underline{C}$ \\
                \hline
              \end{tabular} & \begin{tabular}{|l|}
              \hline
                     $\Gamma \vdash \textbf{fwd}\:f\:p\:k \mapsto c\::\:\underline{C}$ \\
                \hline
              \end{tabular} & \begin{tabular}{|l|}
              \hline
                     $\Gamma \vdash h\::\:\underline{C} \Rightarrow \underline{D}$ \\
                \hline
              \end{tabular} & Handler typering \\
         \end{tabular} \\
         % rules
          \begin{tabular}{c}
          \\
            $\inference{}{\Gamma \vdash .\::\:\underline{C}}[T-Empty]$ \\ 
            \\
            $\inference{\Gamma \vdash oprs \::\: \underline{C} \qquad (l\::\: A_l \rightarrowtriangle B_l) \in \Sigma \qquad \Gamma,\:x\::\:A_l,\:k\::\:B_l \rightarrow \underline{C} \vdash c \::\: \underline{C}}{\Gamma \vdash \textbf{op}\:l\:x\:k \mapsto c,\:oprs\::\: \underline{C}}[T-OprOp]$\\
            \\
            $\inference{\Gamma \vdash oprs\::\:M\:A!\langle E \rangle \qquad (l\::\:A_l \rightarrowtriangle b_l) \in  \Sigma) \\
            \Gamma, \: \beta ,\:x\::\:A_l , \:p\::\:B_l \rightarrow M\: \beta ! \langle E \rangle ,\:k\::\: \beta \rightarrow M\:A! \langle E \rangle \vdash c \::\: M\:A! \langle E \rangle}{\Gamma \vdash \textbf{sc}\:l\:v\:p\:k \mapsto c,\:oprs\::\:M\:A! \langle E \rangle}[T-OprSc]$\\
            \\
            $\inference{A_p = \alpha \rightarrow M\: \beta ! \langle E \rangle \qquad A_k = \beta \rightarrow M \: A!\langle E \rangle \qquad A_k' = M\: \beta \rightarrow M\:A!\langle E \rangle \\
            \Gamma,\: \alpha,\:\beta,\:p\::\:A_p,\:k\::\:A_k,\:f\::\:(A_p,\:A_k') \rightarrow M\:A!\langle E \rangle \vdash c_f \::\: M\:A!\langle E \rangle}{\Gamma \vdash \textbf{fwd}\:f\:p\:k \mapsto c_f \::\: M\:A!\langle E \rangle}[T-Fwd]$\\
            \\
            $\inference{\langle E \rangle = \langle labels \: (oprs);\:F \rangle \qquad \Gamma, \: \alpha \vdash \textbf{return} \   x \mapsto c_r \::\: M\:\alpha ! \langle F \rangle \\ \Gamma,\:\alpha \vdash oprs\::\: M \: \alpha !\langle F \rangle \qquad \Gamma,\: \alpha \vdash \textbf{fwd}\:f\:p\:k \mapsto c_f\::\:M\:\alpha ! \langle F \rangle}{\Gamma \vdash \textbf{handler}\:\{\:\textbf{return},\:oprs,\:\textbf{fwd}\:\}\::\:\forall\:\alpha.\:\alpha!\langle E \rangle \Rightarrow M\:\alpha ! \langle F \rangle}[T-Handler]$\\
            \\
          \end{tabular} \\
          % bottom
          \\
        \hline
    \end{tabular}
    \caption{Handler typering van $\lambda_{sc}$}
    \label{fig:typeHand}
\end{table}

Tabel \ref{fig:typeHand} toont de typeringsregels voor handlers in $\lambda_{sc}$. In deze typering is onderscheid te maken tussen \emph{operatie clausules}, \emph{forwarding clausules} en \emph{handler} typering. \newline 
De operatie clausules worden geven door \textbf{T-Empty} voor de \emph{lege operatie sequence}, \textbf{T-OprOp} voor een \emph{algebraïsche operatie} en \textbf{T-OprSc} voor een \emph{scoped operatie}. Merk op dat de operaties allemaal in type $\underline{C}$ moeten overeenkomen. \newline
De \emph{forwarding clausule} wordt getypeerd door \textbf{T-Fwd}. \newline
De \emph{handler typering} door \textbf{T-Hand} specifieert dat de handler bestaat uit een \emph{\textbf{return} clausule, geen of meerdere operaties clausules en een forwarding clausule}. 

\subsection{Metatheorie}
De belangrijkste metatheoretische eigenschap van $\lambda_{sc}$-calculus is dat deze \emph{typeveilig} is. Dit wordt aangetoond door een bewijs van \emph{progress} en van \emph{subject reduction of preservation}. Deze bewijzen zijn uitgeschreven in de paper ''A Calculus for Scoped Effects \& Handlers'' \cite{Bosman2022} maar worden hier niet verder besproken maar het is belangrijk in te zien dat de toevoegingen de typeveiligheid niet in gevaar brengen.

\subsection{Observaties over de calculus}
Een typerend kenmerk van de calculus is dat de calculus een \emph{fine-grained call-by-value} evaluatie modelleert. \emph{De strikte scheiding van waarden en computaties en reductie op de computaties} vormt eveneens een significant verschil tegenover formele systemen in andere bronnen zoals Types and Programming Languages\cite{Pierce2002}. \newline
Binnen het effect handler paradigma zijn de grootste veranderingen van de $\lambda_{sc}$-calculus tegenover een calculus voor algebraïsche effecten zoals Eff \cite{Bauer2015} enerzijds de introductie van de \textbf{sc} operatie voor scoped effecten en de syntaxis en semantiek hierrond, inclusief de nood voor niet-generieke forwarding doormiddel van de \textbf{fwd} clausule. Om dit te bereiken is de introductie van let-polymorfisme nodig die de syntactische toevoeging van het \textbf{let} sleutelwoord noodzaakt. Het \textbf{let} sleutelwoord laat toe om polymorfe handlers te introduceren, welke nodig zijn om scoped operaties te typeren, specifiek de scoped computatie binnen de scoped effect clausule.

\section{\texorpdfstring{$\lambda^{p}:$}{} Parallelle algebraïsche effect handlers} \label{hoofdstuk:startpuntParallel}
% TODO: grote en kleine stappen op juiste manier presenteren
Deze sectie presenteert een calculus voor de \emph{parallelle behandeling van algebraïsche effecten}. De calculus hieronder gepresenteerd is een lichte aangepaste versie van de calculus beschreven in "Parallel Algebraic Effect Handlers" \cite{Xie2021}. De aanpassingen aan de calculus zijn bedoeld als \emph{refactoring} zodat de vorm dichter aanleunt bij de vorm van de andere calculi gepresenteerd in deze thesis maar de \emph{functie bewaard blijft} zoals in de paper. Merk op dat de calculus een grote-staps operationele semantiek modelleert in tegenstelling tot de andere calculi in deze thesis en dat de calculus ongetypeerd is.

\subsection{Syntaxis}
\begin{table}
    \centering
    \begin{tabular}{|r c l r|}
    \hline
         & & & \\ 
         waarden $v,\:f,\:n$ & $::=$ & $i \: \: | \: \: x \: \: | \: \: \lambda \:x.\:e \: \:$ & \\
         & $|$ & $\langle v_{0},\: ... , \: v_{n} \rangle \: \: | \: \: \textbf{perform}\:op$ & \\
         & & & \\
         handlers $h$ & $::=$ & $\{ \: \: \textbf{return} \mapsto f_{r}$ & return clausule\\
         & & $, \: \textbf{op} \mapsto f_{p}$ & algebraïsche effect  clausule \\
         & & $, \: \textbf{traverse} \mapsto f_{t} \: \} $ & traverse clausule \\
         & & & \\
         expressies $e$ & $::=$ & $v$ & waarde \\
          & $|$ & $e\:e$ & applicatie \\
          & $|$ & $\textbf{for}\:x\::\:n.\:e$ & for constructie \\
          & $|$ & $\textbf{handle}\:h\:e$ & behandel frame \\
         & & & \\
         evaluatie context $F$ & $::=$ & $. \: \: | \: \: F \: e \: \: | \: \: v\:F$ & parallelle context \\
         $E$ & $::=$ & $. \: \: | \:\: E\:e \: \: | \: \: v\:E \:\: | \:\: \textbf{handle}\:h\:e$ & sequentiële context\\
         & & & \\
    \hline
    \end{tabular}
    \caption{Syntaxis voor $\lambda^{p}$}
    \label{fig:syntaxisPar}
\end{table}

Tabel \ref{fig:syntaxisPar} geeft de syntaxis voor de $\lambda^{p}$-calculus weer. \newline
Waar de $\lambda_{sc}$-calculus de termen verdeelt in waarden en computaties, verdeelt deze calculus de \emph{termen in expressies en handlers}. De verschillen tegenover de eerder besproken calculus zijn dat \emph{waarden expressies zijn} en \emph{handlers niet langer waarden zijn} maar een andere categorie die zich buiten de expressies bevindt. \newline
De \textbf{waarden} zijn ofwel een \emph{literal} ($i$), een \emph{variabele} ($x$), een \emph{lambda functie} ($\lambda\:x.\:e$), een \emph{lijst van waarden} ($\langle v_{0},\: ...,\: v_{n} \rangle$) of een \emph{oproep voor een algebraïsch effect} ($\textbf{perform}\:op$). Voor waarden wordt $v$ gebruikt voor \emph{algemene waarden}, $f$ voor \emph{lambda-functies} en $n$ voor \emph{literals}. \newline 
\textbf{Handlers} hebben steeds 3 clausules, namelijk een \emph{return clausule} ($\textbf{return} \mapsto f_{r}$), een \emph{algebraïsche effect clausule} ($\textbf{op} \mapsto f_{p}$) en een \emph{traverse clausule} ($\textbf{traverse} \mapsto f_{t}$). De \textbf{traverse} clausule is essentieel in de parallelle behandeling. \newline 
\textbf{Computaties} zijn ofwel een \emph{waarde} ($v$), een \emph{applicatie} ($e \  e$), een \emph{for constructie} ($\textbf{for} \  x \ \  n.\  e$) voor parallelle behandeling of een \emph{handle constructie} ($\textbf{handle} \  h  \  e$) voor behandeling van een expressies door een handler. \newline
Opvallend aan de syntaxis van handlers is dat elke handler precies \'{e}\'{e}n clausule heeft van elk type en de clausule hebben geen label.

\subsection{Operationele semantiek}

\begin{table}
    \centering
    \begin{tabular}{|l|}
        \hline
        % top
        \\
        % header
         \begin{tabular} {l r}
              \begin{tabular}{|l|}
              \hline
                     $e \leadsto e'$ \\
                \hline
              \end{tabular} & Reductie van expressies \\
         \end{tabular} \\
         % rules
          \begin{tabular}{c}
            $\inference{}{(\lambda\:x.\:e)\:v \leadsto e\:[\:v\:/\:x\:]}[E-App]$ \\ 
            \\
            $\inference{}{\langle v_{0},\: ... ,\: v_{n} \rangle \: i \leadsto v_{i}}[E-Index]$ \\
            \\
            $\inference{(\textbf{return} \mapsto f_{r}) \in h}{\textbf{handle}\:h\:v \leadsto f_{r}\:v}[E-Return]$ \\
            \\
            $\inference{\textbf{op} \notin bop(E) \wedge (\textbf{op} \mapsto f_{p}) \in h \\ where\:k=\:\lambda\:x.\:\textbf{handle}\:h\:E[x]}{\textbf{handle}\:h\:E[\textbf{perform}\:op\:v] \leadsto f_{p}\:v\:k}[E-Perform]$ \\
            \\
            $\inference{(\textbf{traverse} \mapsto f_{t}) \in h \\
            where\:l\:=\:\textbf{for}\:x\::\:n.\:\textbf{handle}\:h\:e \\
            k\:=\:\lambda\:xs.\:\textbf{handle}\:h\:F[xs]}{\textbf{handle}\:h\:F[\textbf{for}\:x\::\:n.\:e] \leadsto f_{t}\:n\:l\:k}[E-Traverse]$ \\
            \\
            $\inference{e \leadsto e'}{E[e] \mapsto E[e']}[E-Step]$ \\
            \\
            $\inference{\forall \: 0 \: \leq \: i < n. \  e[x\:::=\:i] \mapsto v_{i}}{F[\textbf{for}\:x\::\:n. \  e] \mapsto F[\langle v_{0},\:...,\:v_{n-1}\rangle]}[E-Parallel]$ \\
          \end{tabular} \\
          % bottom
          \\
        \hline
    \end{tabular}
    \caption{Operationele semantiek van $\lambda^{p}$}
    \label{fig:semantiekPar}
\end{table}

De operationele semantiek in deze calculus is afgebeeld in Tabel \ref{fig:semantiekPar}. Merk op dat deze operationele semantiek niet volledig is. Het is \emph{niet mogelijk} om puur op basis van de reductie-regels een expressie te reduceren tot een waarde omdat de calculus geen reducties definieert voor pure expressies. De calculus modelleert een combinatie van fijne stappen met de $\leadsto$ relatie en grote stappen met de $\mapsto$ relatie. \newline 
De regels \textbf{E-App}, \textbf{E-Index} zijn standaard regels voor \emph{applicatie} en \emph{lijst-extractie}. \newline
De \textbf{E-Step} is een \emph{congruentie regel voor evaluatie binnen de sequentiële evaluatie context} ($E$). De evaluatie stap is hier een individuele stap. \newline 
\textbf{E-Return} past de \emph{return clausule} ($(\textbf{return} \mapsto f_{r}) \in h$) in de handler toe wanneer de handler een waarde tegenkomt. \newline
\textbf{E-Perform} past een \emph{algebraïsche operatie clausule} toe wanneer een handler een effect tegenkomt dat in de handler zit ($(\textbf{op} \mapsto f_{p}) \in h$).\newline %TODO: E-Traverse en E-Parallel uitleg 
De meest \emph{essentiële regel voor de werking} van de $\lambda^{p}$-calculus is \textbf{E-Traverse}. Deze regel modelleert het effectvol behandelen door de handler van de \textbf{for}-constructie die geïntroduceerd wordt door de paper. In essentie vervangt deze regel een \textbf{for}-constructie in het programma door een handler-specifieke computatie $f_t$. Deze computatie kan vervolgens geen tot meerdere nieuwe \textbf{for}-constructies introduceren. Bij deze behandeling schuift de handler door in de continuatie en de computatie $f_t$ wat \emph{diepe behandeling} toelaat. \newline
\textbf{E-Parallel} is de regel waarin het parallellisme in de calculus plaatsvindt. De regel evalueert een pure \emph{\textbf{for}-constructie in parallel naar een lijst van waarden.} De computatie $e$ wordt in parallel toegepast voor elke $x$ in $n$. \emph{Merk op dat in deze regel de stap, voor elke i in n, een grote stap ($\mapsto v_i$) is die de expressie in \'e\'en keer naar een waarde evalueert}. Deze regel behandelt de pure \textbf{for} constructie zonder handler erbuit en zou vrij moeten zijn van oproepen naar effecten. \newline

\subsection{Observaties over de calculus}
De $\lambda^p$-calculus modelleert geen volledig uitgewerkte operationele semantiek aangezien stappen voor \emph{pure computatie} ontbreken net zoals regels voor \emph{forwarding} van handlers. De calculus gebruikt $e \leadsto e'$ gebruikt wordt voor zowel individuele stappen als voor een volledige reductie naar een waarde. Het gebruik in \textbf{E-Step} suggereert een individuele stap maar het gebruik \textbf{E-Parallel} suggereert een volledige reductie. \newline
Opvallend is dat de calculus geen type- en effect-systeem modelleert. \newline
De E-Parallel is mogelijk uitdagend te implementeren in een interpreter. Hoe kan de interpreter efficiënt detecteren wanneer geen handlers meer buiten de \textbf{for} constructie resteren?
%Doordat elke handler in de \textbf{E-Traverse} clausule nieuwe \textbf{for} constructies kan introduceren is het nodig om als buitenste handler het programma te evalueren met een handler die de resterende \textbf{for}-constructies puur behandelt. 

\section{Gecombineerde calculus}
% TODO: 
De $\lambda_{sc}$-calculus ondersteunt de sequentiële behandeling van algebraïsche en scoped effecten. De $\lambda^{p}$-calculus ondersteunt de parallelle afhandeling van algebraïsche effecten. Het doel van de masterproef is een gecombineerde calculus te bekomen die de combinatie van beide ondersteunt. Uit dit hoofdstuk blijkt dat er significante verschillen zijn in het ontwerp van beide calculi. \newline 
Hoofdstuk \ref{hoofdstuk:motivatie} bespreekt wat het plan van aanpak is om beide calculi samen te voegen en wat de uitdagingen zijn. De gecombineerde calculus bouwt verder op de $\lambda_{sc}$-calculus en probeert de bevindingen uit de $\lambda^p$-calculus hierin in te passen. \newline
Hoofdstukken \ref{hoofdstuk:syntaxis} en \ref{hoofdstuk:semantiek} behandelen de syntaxis en semantiek voor de gecombineerde calculus $\lambda_{sc}^{p}$. \newline
Hoofdstukken \ref{hoofdstuk:typesysteem} en \ref{hoofdstuk:metatheorie} bespreken vervolgens respectievelijk het type-en-effectsysteem en de metatheorie met een bewijs voor typeveiligheid voor de gecombineerde calculus. Hoofdstuk \ref{hoofdstuk:voorbeelden} behandelt voorbeelden van het gebruik van de gecombineerde calculus.
%Een hoofdstuk behandelt een samenhangend geheel dat min of meer op zichzelf
%staat. Het is dan ook logisch dat het begint met een inleiding, namelijk
%het gedeelte van de tekst dat je nu aan het lezen bent.

%\section{Eerste onderwerp in dit hoofdstuk}
%De inleidende informatie van dit onderwerp.

%\subsection{Een item}
%Een tekst staat nooit alleen. Dit wil zeggen dat er zeker ook referenties
%nodig zijn. Dit kan zowel naar on-line documenten\cite{wiki} als naar
%boeken\cite{pratchett06:_good_omens}.

%\section{Figuren}
%Figuren worden gebruikt om illustraties toe te voegen. Dit is dan ook de
%manier om beeldmateriaal toe te voegen zoals getoond wordt in
%figuur~\ref{fig:logo}.

%\begin{figure}
%  \centering
%  \includegraphics{logokul}
%  \caption{Het KU~Leuven logo.}
%  \label{fig:logo}
%\end{figure}

%\section{Tabellen}
%Tabellen kunnen gebruikt worden om informatie op een overzichtelijke te
%groeperen. Een tabel is echter geen rekenblad! Vergelijk maar eens
%tabel~\ref{tab:verkeerd} en tabel~\ref{tab:juist}. Welke tabel vind jij het
%duidelijkst?

%\begin{table}
%  \centering
%  \begin{tabular}{||l|lr||} \hline
%    gnats     & gram      & \$13.65 \\ \cline{2-3}
%              & each      & .01 \\ \hline
%    gnu       & stuffed   & 92.50 \\ \cline{1-1} \cline{3-3}
%    emu       &           & 33.33 \\ \hline
%    armadillo & frozen    & 8.99 \\ \hline
%  \end{tabular}
%  \caption{Een tabel zoals het niet moet.}
%  \label{tab:verkeerd}
%\end{table}

%\begin{table}
%  \centering
%  \begin{tabular}{@{}llr@{}} \toprule
%    \multicolumn{2}{c}{Item} \\ \cmidrule(r){1-2}
%    Animal    & Description & Price (\$)\\ \midrule
%    Gnat      & per gram    & 13.65 \\
%              & each        & 0.01 \\
%    Gnu       & stuffed     & 92.50 \\
%    Emu       & stuffed     & 33.33 \\
%    Armadillo & frozen      & 8.99 \\ \bottomrule
%  \end{tabular}
%  \caption{Een tabel zoals het beter is.}
%  \label{tab:juist}
%\end{table}

%\section{Lorem ipsum}
%Tenslotte gaan we hier nog wat tekst voorzien zodat er minstens een
%bijkomende bladzijde aangemaakt wordt. Dat geeft de gelegenheid om eens te
%zien hoe de koptekst en de voettekst zich gedragen.

%\subsection{Lorem ipsum dolor sit amet, consectetur adipiscing elit}
%Sed nec tortor id felis tristique sodales. Nulla nec massa eu dui fermentum
%tincidunt. Integer ullamcorper ante eget eros posuere faucibus. Nam id
%ligula ut augue pulvinar vulputate id at purus. Aenean condimentum tortor
%eu mi placerat eget eleifend massa mollis. Nam est mi, sagittis quis
%euismod eget, sagittis in nibh. Proin elit turpis, aliquam et imperdiet
%sed, volutpat eu turpis.

%Pellentesque vel enim tellus, vitae egestas turpis. Praesent malesuada elit
%non nisi sollicitudin non blandit lacus tincidunt. Morbi blandit urna at
%lectus ornare laoreet. Suspendisse turpis diam, lobortis dictum luctus
%quis, commodo at lorem. Integer lacinia convallis ultricies. Sed quis augue
%neque, eu malesuada arcu. Nullam vehicula, purus vitae sagittis pulvinar,
%erat eros semper massa, eu egestas nibh erat quis magna. Cras pellentesque,
%nisl eu dapibus volutpat, urna augue ornare quam, quis egestas lectus nulla
%a lectus.

%Vivamus dictum libero in massa cursus sed vulputate eros imperdiet. Donec
%lacinia, libero ac lobortis egestas, nibh dui ornare arcu, luctus porttitor
%velit massa sit amet quam. Maecenas scelerisque laoreet diam, vitae congue
%quam adipiscing vitae. Aliquam cursus nisl a leo convallis eleifend
%fermentum massa porta. Nunc libero quam, dapibus dapibus molestie sit amet,
%faucibus vel nunc.

%\subsection{Praesent auctor venenatis posuere}
%Sed tellus augue, molestie in pulvinar lacinia, dapibus non ipsum. Fusce
%vitae mi vitae enim ullamcorper hendrerit eu malesuada est. Proin iaculis
%ante sed nibh tincidunt vel interdum libero posuere. Vivamus accumsan metus
%quis felis congue suscipit dapibus enim mattis. Fusce mattis tortor eget
%ipsum interdum sagittis auctor id metus.

%Integer diam lacus, pharetra sit amet tempor et, tristique non lorem.
%Aenean auctor, nisi eu interdum fermentum, lectus massa adipiscing elit,
%sed facilisis orci odio a lectus. Proin mi nibh, tempus quis porta a,
%viverra quis enim. In sollicitudin egestas libero, quis viverra velit
%molestie eget. Nulla rhoncus, dolor a mollis vestibulum, lacus elit semper
%nisi, nec sollicitudin sem urna eu magna. Nunc sed est urna, euismod congue
%mi.

%\subsection{Cras vulputate ultricies venenatis}
%Vivamus eros urna, sodales accumsan semper vel, lobortis sit amet mauris.
%Etiam condimentum eleifend lorem, ullamcorper ornare lectus aliquet vitae.
%Praesent massa enim, interdum sit amet semper et, venenatis ut elit.
%Quisque faucibus, quam ac lacinia imperdiet, nulla neque elementum purus,
%tempus rutrum justo massa porta sapien. Vestibulum ante ipsum primis in
%faucibus orci luctus et ultrices posuere cubilia Curae; Sed ultrices
%interdum mi, et rhoncus sapien rutrum sed.

%Duis elit orci, molestie quis sollicitudin sed, convallis non ante.
%Maecenas tincidunt condimentum justo, et ultricies leo tristique vitae.
%Vestibulum quis quam non lectus dapibus eleifend a vitae nibh. Nam nibh
%justo, pharetra quis iaculis consequat, elementum quis justo. Etiam mollis
%lacinia lacus, nec sollicitudin urna lobortis ac. Nulla facilisi.

%Proin placerat risus eleifend erat ultricies placerat. Etiam rutrum magna
%nec turpis euismod consectetur. Phasellus tortor odio, lacinia imperdiet
%condimentum sed, faucibus commodo erat. Phasellus sed felis id ante
%placerat ultrices. Aenean tempor justo in tortor volutpat eu auctor dolor
%mollis. Aenean sit amet risus urna. Morbi viverra vehicula cursus.

%\subsection{Donec nibh ante, consectetur et posuere id, tempus nec arcu}
%Curabitur a tellus aliquet ipsum pellentesque scelerisque. Etiam congue,
%risus et volutpat rutrum, est purus dapibus leo, non cursus metus felis
%eget ligula. Vivamus facilisis tristique turpis, ut pretium lectus luctus
%eleifend. Fusce magna sapien, ullamcorper vitae fringilla id, euismod quis
%ante.

%Phasellus volutpat, nunc et pharetra semper, sem justo adipiscing mauris,
%id blandit magna quam et orci. Vestibulum a erat purus, ut molestie ante.
%Vestibulum ante ipsum primis in faucibus orci luctus et ultrices posuere
%cubilia Curae; Proin turpis diam, consequat ut ullamcorper ut, consequat eu
%orci. Sed metus risus, fringilla nec interdum vel, interdum eu nunc.
%Suspendisse vel sapien orci.

%\subsection{Morbi et mauris tempus purus ornare vehicula}
%Mauris sit amet diam quam, eget luctus purus. Sed faucibus, risus semper
%eleifend iaculis, mi turpis bibendum nisl, quis cursus nibh nisl sit amet
%ipsum. Vestibulum tempor urna vitae mi auctor malesuada eget non ligula.
%Nullam convallis, diam vel ultrices auctor, eros eros egestas elit, sed
%accumsan arcu tortor eget leo. Vestibulum orci purus, porttitor in pharetra
%eget, tincidunt eget nisl. Nullam sit amet nulla dui, facilisis vestibulum
%dui.

%Donec faucibus facilisis mauris ac cursus. Duis rhoncus quam sed nisi
%laoreet eu scelerisque massa tincidunt. Vivamus sit amet libero nec arcu
%imperdiet tempor quis non libero. Sed consequat dignissim justo. Phasellus
%ullamcorper, velit quis posuere vulputate, felis erat tincidunt mauris, at
%vestibulum justo lectus et turpis. Maecenas lacinia convallis euismod.
%Quisque egestas fermentum sapien eu dictum. Sed nec lacus in purus dictum
%consequat quis vel nisl. Fusce non urna sem. Curabitur eu diam vitae elit
%accumsan blandit. Nullam fermentum nunc et leo dictum laoreet. Donec semper
%varius velit vel fringilla. Vivamus eu orci nunc.

%\section{Besluit van dit hoofdstuk}
%Als je in dit hoofdstuk tot belangrijke resultaten of besluiten gekomen
%bent, dan is het ook logisch om het hoofdstuk af te ronden met een
%overzicht ervan. Voor hoofdstukken zoals de inleiding en het
%literatuuroverzicht is dit niet strikt nodig.

%%% Local Variables: 
%%% mode: latex
%%% TeX-master: "masterproef"
%%% End: 

\chapter{Motivatie en Uitdagingen}
\label{hoofdstuk:motivatie}
Dit hoofdstuk behandelt de motivatie voor dit onderwerp en de uitdagingen die vermeldenswaardig zijn bij de zoektocht naar een gepaste oplossing.

\section{Sequentiële vs parallelle uitvoering}

\begin{figure}
    \centering
    \includegraphics[width=\textwidth]{Media/scope.png}
    \caption{Scope van de verschillende calculi vermeld in deze thesis}
    \label{fig:motiv}
\end{figure}

De primaire motivatie voor de ontwikkeling van de calculus is het maken van een calculus die zowel parallelle als sequentiële uitvoering van zowel algebraïsche als scoped effecten ondersteunt. De $\lambda_{sc}$-calculus \cite{Bosman2022} ondersteunt sequentiële uitvoering van algebraïsche en scoped effecten. De $\lambda^{p}$-calculus ondersteunt parallelle uitvoering van algebraïsche effecten. Het doel van de $\lambda_{sc}^{p}$-calculus is om vrije keuze te geven over parallelle of sequentiële uitvoering van algebraïsche of scoped effecten zoals afgebeeld in Figuur \ref{fig:motiv}. 

\subsection{Aanpak}
Omdat parallelle of sequentiële uitvoering een verschil in semantiek geeft en in de effect handler aanpak de semantiek doorgeschoven is naar de handler, gebeurt dit ook in de $\lambda_{sc}^{p}$-calculus voor de semantiek van uitvoeringswijze. Concreet zal elke handler een clausule uitwerken die bepaalt hoe een lijst van computatie met de effecten die de handler behandelt, behandeld worden.

\subsection{Voorbeeld} 
\begin{equation} \label{eq:motivEx}
    \begin{split}
        \textbf{for}\:(1:2:3:4:5:[\:\:])\:(x.\:\textbf{if}\:x \equiv 2 & \\
        & \textbf{then}\:\textbf{op}\:throw\:''error''\:(x.\:\textbf{return}\:x);\\
        & \textbf{else}\:\textbf{op}\:accum\:x\:(x.\:\textbf{return}\:x))\:(x.\:\textbf{return}\:x) \\
    \end{split}
\end{equation}

Het voorbeeld programma in Eq. \ref{eq:motivEx} is een programma dat een accumulatie doet in een string waarbij een functie mapt over een lijst met getallen van 1 tot 5. Dit is een voorbeeld uit \cite{Xie2021} met herwerkte syntaxis. De functie gooit een fout als de input 2 is, anders voegt de functie de input bij het resultaat. Bij sequentiële uitvoering zouden we verwachten dat het programma de executie stopt vanaf het de functie met input 2 behandelt. Het resultaat van de accumulatie is dan "1". Bij parallelle afhandeling gaat de exception bij input 2 geen effect kunnen hebben op de andere accumulaties die parallel gebeuren en zal het resultaat "1345" zijn. Deze verschillende semantiek zou kunnen bekomen worden door 2 verschillende handlers te maken die het exception effect afhandelen waarbij de ene werkt op de sequentiële wijze en de andere op de parallelle wijze.

%Een hoofdstuk behandelt een samenhangend geheel dat min of meer op zichzelf
%staat. Het is dan ook logisch dat het begint met een inleiding, namelijk
%het gedeelte van de tekst dat je nu aan het lezen bent.

%\section{Eerste onderwerp in dit hoofdstuk}
%De inleidende informatie van dit onderwerp.

%\subsection{Een item}
%De bijbehorende tekst. Denk eraan om de paragrafen lang genoeg te maken en
%de zinnen niet te lang.

%Een paragraaf omvat een gedachtengang en bevat dus steeds een paar zinnen.
%Een paragraaf die maar \'e\'en lijn lang is, is dus uit den boze.

%\section{Tweede onderwerp in dit hoofdstuk}
%Er zijn in een hoofdstuk verschillende onderwerpen. We zullen nu
%veronderstellen dat dit het laatste onderwerp is.

%\section{Besluit van dit hoofdstuk}
%Als je in dit hoofdstuk tot belangrijke resultaten of besluiten gekomen
%bent, dan is het ook logisch om het hoofdstuk af te ronden met een
%overzicht ervan. Voor hoofdstukken zoals de inleiding en het
%literatuuroverzicht is dit niet strikt nodig.

%%% Local Variables: 
%%% mode: latex
%%% TeX-master: "masterproef"
%%% End: 

\chapter{Syntaxis}
\label{hoofdstuk:syntaxis}
% TODO: andere syntaxis voor lege lijst waarden en computaties
\begin{table}
    \centering
    \begin{tabular}{|r c l r|}
    \hline 
    & & & \\
         waarden $v$ & $::=$ & $() \: \: | \: \: (v_{1}, \: v_{2} ) \: \: | \: \: x \: \: | \: \: \lambda x . \: c \: \: | \: \: h$ & \\
         & $|$ & \hl{lstv} & \\
         handlers $h$ & $::=$ & $\textbf{handler} \: \{ \: \: \textbf{return} \: x \mapsto c_{r}$ & return clausule\\
         & & $\qquad \qquad \quad , \: oprs$ & effect  clausules \\
         & & $\qquad \qquad \quad , \: \textbf{fwd} \: f \: p \: k \mapsto c_{f} \}$ & forwarding clausule \\
         \hl{lijst waarden $lstv$} & $::=$ & \hl{$[\:\:]\:\:|\:\:v:lstv$} & \hl{lijst van waarden}\\
          effect clausules $oprs$ & $::=$ & . & \\ 
          & $|$ & $\textbf{op} \: l \: x \: k \mapsto c, \: oprs$ & algebraïsch effect clausule\\
           & $|$ & $\textbf{sc} \: l \: x \: p \: k \mapsto c, \: oprs$ & scoped effect clausule\\
        & $|$ & \hl{$\textbf{for}\:l\:lstv\:p\:k \mapsto c, \  oprs$} & \hl{for effect clausule} \\
        & & & \\
         computaties $c$ & $::=$ & $\textbf{return} \: v$ & return waarde \\
          & $|$ & $\textbf{op} \: l \: v \: (y. \: c)$ & algebraïsch effect \\
          & $|$ & $\textbf{sc} \: l \: v \: (y. \: c_{1}) \: (z. \: c_{2})$ & scoped effect \\
          & $|$ & \hl{$\textbf{for} \  l \: v \: (y. \: c_{1}) \: (z. \: c_{2})$} & \hl{for effect} \\
          & $|$ & $v \star c$ & behandeling \\
          & $|$ & $\textbf{do} \: x \leftarrow c_{1}\:; \: c_{2}$ & do clausule \\
          & $|$ & $v_{1} \: v_{2}$ & applicatie \\
          & $|$ & $\textbf{let} \: x = v \: \textbf{in} \: c$ & let \\
          & $|$ & \hl{\textbf{head}\:v} & \hl{head lstv} \\
          & $|$ & \hl{\textbf{tail}\:v} & \hl{tail lstv} \\
          & $|$ & \hl{\textbf{empty}\:v} & \hl{test lege lstv} \\
          & $|$ & \hl{\textbf{head}\:c} & \hl{head lstc} \\
          & $|$ & \hl{\textbf{tail}\:c} & \hl{tail lstc} \\
          & $|$ & \hl{\textbf{empty}\:c} & \hl{test lege lstc} \\
          & $|$ & \hl{$lstc$} & \hl{lijst computaties} \\
          \hl{lijst computaties $lstc$} & $::=$ & \hl{$[\:\:]\:\:|\:\:c\::\:lstc$} & \hl{lijst computaties}\\
         & & & \\
         waarde types $A, \: B, \: M$ & $::=$ & $() \: \: | \: \: (A, \:B) \: \: | \: \: A \rightarrow \underline{C} \: \: | \: \: \underline{C} \Rightarrow \underline{D}$ & \\
         & $|$ & $\alpha$ & type variabele \\
         & $|$ & $\lambda \: \alpha . \: A$ & type operator abstractie \\
         & $|$ & $M \: A$ & type applicatie \\
         type schemas $\sigma$ & $::=$ & $A \: \: | \: \: \forall \: \mu . \: \sigma \: \: | \: \: \forall \: \alpha. \: \sigma $ & \\
         computatie types $\underline{C}, \: \underline{D}$ & $::=$ & $A ! \langle E \rangle $ & \\
         effect type rows $E, \: F$ & $::=$ & $. \: \: | \: \: \mu \: \: | \: \: l; \: E $ & \\
         kinds $K$ & $::=$ & $* \: \: | \: \: K \rightarrow K$ & \\
         signatuur contexten $\Sigma$ & $::=$ & $. \: \: | \: \: \Sigma , \: l \: : \: A \rightarrowtriangle B$ & \\
         type contexten $\Gamma$ & $::=$ & $. \: \: | \:\: \Gamma, \: x \: : \: A \: \: | \: \: \Gamma , \: \mu \: \: | \: \: \Gamma, \: \alpha $ & \\
         & & & \\
    \hline
    \end{tabular}
    \caption{$\lambda_{sc}^{p}$ Syntaxis met uitbreidingen gemarkeerd}
    \label{fig:syntaxisNodig}
\end{table}


\begin{table}
    \centering
    \begin{tabular}{|r c l r|}
    \hline 
        & & & \\
        waarden $v$ & $::=$ & \hl{true | false} & \hl{booleans} \\
        & & & \\
         computaties $c$ & $::=$ & \hl{$\textbf{let rec} \: f \: = \: c_{1} \: \textbf{in} \: c_{2}$} & \hl{let rec} \\
         & $|$ & \hl{\textbf{if} v \textbf{then} $c_1$ \textbf{else} $c_2$} & \hl {conditie} \\
          & $|$ & \hl{\textbf{fst}\:v} & \hl{first paar} \\
          & $|$ & \hl{\textbf{snd}\:v} & \hl{second paar} \\
          & $|$ & \hl{\textbf{map}\:f\:v} & \hl{map over lstv} \\
          \\
    \hline
    \end{tabular}
    \caption{Uitbreidingen op de $\lambda_{sc}^{p}$ Syntaxis om de calculus meer bruikbaar te maken}
    \label{fig:syntaxisHandig}
\end{table}

Dit hoofdstuk bespreekt de syntaxis voor de gecombineerde $\lambda_{sc}^p$-calculus. Tabel \ref{fig:syntaxisNodig} stelt de syntaxis voor met de \emph{nodige uitbreidingen} op de $\lambda_{sc}$-calculus gemarkeerd. De belangrijkste toevoeging is de \textbf{for} clausule in de handler. Bijhorend is een lijst-structuur voor waarden en een lijst-structuur voor computaties. Tabel \ref{fig:syntaxisHandig} toont \emph{handige uitbreidingen} om de evaluatie van meer programma's mogelijk te maken. Deze tabel beschrijft een \textbf{let rec} constructie om recursieve functie evaluatie mogelijk te maken, een \textbf{if} conditionele constructie en enkele hulp-functies om paar- en lijst-manipulatie toe te laten. Deze toevoegingen zijn niet essentieel om een minimaal werkende calculus te bekomen maar wel nodig om de voorbeelden te kunnen evalueren. De volgende secties bespreken de verschillende toegevoegde syntactische elementen.

\section{For clausule}
De \textbf{for} clausule modelleert parallelle behandeling in de calculus. Het doel is om zoals in de $\lambda^p$-calculus uitgaande van een parallelle pure \emph{for each} functie een parallelle, gebruiker-gedefenieerde parallelle effecten te modelleren\cite{Xie2021}. \newline 
De \textbf{for} clausule die deze syntaxis aanlevert, bestaat uit het sleutelwoord \textbf{for}, gevolgd door een label,  een lijst van waarden, gevolgd door een functie $(y. \ 
 c_{1})$, gevolgd door de resumptie $(z. \  c_{2})$ die de continuatie van het programma bevat. Merk op dat hoewel de syntaxis van de \textbf{for} clausule sterk gelijkt op die van de \textbf{sc} clausule, deze niet als \'e\'en clausule kan beschouwd worden omdat de semantiek zal verschillen. De syntaxis lijkt sterk op de syntaxis uit de $\lambda^p$-calculus met de uitzondering van het toegevoegde label. Dit laat toe om meerdere verschillende parallelle effecten binnen hetzelfde programma te evalueren. \newline 
 %De standaard \textbf{for} behandeling (om een parallelle \emph{for each} constructie te benaderen) zou zijn om de functie, in parallel, op elke waarde in de lijst toe te passen en het resultaat als input van de continuatie te gebruiken. Aangezien de calculus een effect handler aanpak voorstelt, is dat slechts een mogelijke implementatie van de handler. De handler bepaalt hoe de \textbf{for} constructie te behandelen. Aangezien deze computatie nieuwe \textbf{for} constructies kan introduceren, is het raadzaam om elk programma met \textbf{for} constructies als buitenste handler te behandelen door een pure handler om eventuele overgebleven \textbf{for} constructies puur af te handelen.
 %De \textbf{for} clausule heeft in tegenstelling tot de \textbf{op} of \textbf{sc} clausule geen label omdat deze clausule geen specifiek effect voorsteld maar een generieke \textbf{for-each} constructie over een lijst. 

%\begin{equation}
%    \textbf{for}\quad (input_1:input_2:input_3:[\:\:]) \quad (input.\:\textbf{effect}\:input) \quad(result.\:\textbf{rest})
%\end{equation}


%\begin{equation}
%    \begin{split}
%    & \textbf{hPure} = \textbf{handler} \\
%    & \{ \\
%    & \qquad ... \\
%    & \qquad    \textbf{for}\:lst\:p\:k \rightarrow \textbf{do}\:res \leftarrow map\:p\:lst \\
%    &  \qquad \qquad \qquad \qquad k\:res \\
%    & \} 
%\end{split}  
%\end{equation}


\section{Lijst-structuur}
De minimale vereiste om computaties in parallel uit te voeren is een datastructuur waarop een parallelle iterator kan gedefenieerd worden om de verschillende inputwaarden te extraheren.
%samengestelde data-structuur in de calculus die de verschillende computaties die de machine parallel uitvoert bijhoudt. 
Deze calculus gebruikt een lijst-structuur maar in principe is dit mogelijk op elke lineaire datastructuur. \newline
%aangezien op elk van deze het mogelijk is (maar niet noodzakelijk efficiënt) een head, tail en empty functie te implementeren. 
%Een voorbeeld van een lijst van computaties is: 
%\begin{equation} \label{eq:listEx}
%    \begin{split}
%    (39 + 49):(29 + 74):(100 + 61):(41 + 56):(97 + 67):[\:\:] \\
%    \leadsto 88:103:161:97:164:[\:\:]
%    \end{split}
%\end{equation} 
%De reductie in dit voorbeeld kan parallel gebeuren voor elk element in de lijst. \newline
Aangezien de $\lambda_{sc}$-calculus syntaxis een strikte scheiding hanteert tussen waarden en termen ligt het voor de hand dat de $\lambda_{sc}^{p}$-calculus eveneens strikt onderscheid maakt tussen \emph{lijsten van waarden} en \emph{lijsten van computaties}.

\subsection{Lijst van waarden}
Een lijst van waarden is zelf een waarde. De syntaxis is intuïtief, de lijst eindigt op een lege lijst $[\:\:]$ en wordt voorgegaan door één of meerdere waarden.

\subsection{Lijst van computaties}
%Een lijst van compuaties is een computatie, dit omdat de elementen in de lijst kunnen reduceren en semantische regels geïntroduceerd worden in Sectie \ref{hoofdstuk:semantiek} om een lijst te reduceren naar een lijst van normaalwaarden. Hiervoor moet de lijst zelf een computatie zijn omdat een reductie in de $\lambda_{sc}^{p}$-calculus een computatie naar een andere computatie mapt. 
De syntaxis voor een lijst van computaties is intuïtief en analoog aan de lijst van waarden, de lijst eindigt op een lege lijst $[\:\:]$ en wordt voorgegaan door één of meerdere computaties.

\subsection{Lijst-manipulatie functies}
De lijst heeft minimaal een \textbf{head}, \textbf{tail} en \textbf{empty} functie nodig om elementen uit de lijst te halen en na te kijken of de lijst leeg is. Deze minimale implementatie is aanwezig in de syntaxis.

\begin{equation}
    \begin{split}
        & \textbf{head}\:(5:3:8:4:[\:\:]) \\
        & \leadsto \textbf{return}\:5
    \end{split}
\end{equation}

\begin{equation}
    \begin{split}
        & \textbf{head}\:((\textbf{double}\:5):(\textbf{double}\:3):(\textbf{double}\:8):(\textbf{double}\:4):[\:\:]) \\
        & \leadsto \textbf{return}\:(\textbf{double}\:5)
    \end{split}
\end{equation}

\begin{equation}
    \begin{split}
        & \textbf{tail}\:(5:3:8:4:[\:\:]) \\
        & \leadsto \textbf{return}\:(3:8:4:[\:\:])
    \end{split}
\end{equation}

\begin{equation}
    \begin{split}
        & \textbf{tail}\:((\textbf{double}\:5):(\textbf{double}\:3):(\textbf{double}\:8):(\textbf{double}\:4):[\:\:]) \\
        & \leadsto \textbf{return}\:((\textbf{double}\:3):(\textbf{double}\:8):(\textbf{double}\:4):[\:\:]) 
    \end{split}
\end{equation}

\begin{equation}
    \begin{split}
        & \textbf{empty}\:(5:3:8:4:[\:\:]) \\
        & \leadsto \textbf{return}\:false
    \end{split}
\end{equation}

\begin{equation}
    \begin{split}
        & \textbf{empty}\:((\textbf{double}\:5):(\textbf{double}\:3):(\textbf{double}\:8):(\textbf{double}\:4):[\:\:]) \\
        & \leadsto \textbf{return}\:false
    \end{split}
\end{equation}


\begin{equation}
    \begin{split}
        & \textbf{empty}\:[\:\:] \\
        & \leadsto \textbf{return}\:true
    \end{split}
\end{equation}

\subsection{Map functie}
Een map functie die een lijst van waarden omvormt naar een lijst van computaties met als argument een functie om toe te passen op de lijst van waarden.

\begin{equation}
    \begin{split}
        & \textbf{map}\:(+1)\:(5:3:8:4:[\:\:]) \\
        & \leadsto \textbf{return}\:(6:4:9:5:[\:\:])
    \end{split}
\end{equation}

\section{Recursieve let}
\textbf{let rec} helpt om recursieve functies te implementeren in de calculus. De syntaxis is gebaseerd op de syntaxis voor \textbf{let rec} beschreven in hoofdstuk twee van het boek "Principles of Programming Languages" \cite{Palmer2009} en les 11 van CS6110 aan de Cornell-universiteit \cite{Sampson2018}. Voor simpliciteit is gekozen om slechts 1 recursieve functie toe te laten omdat die semantische regels vereenvoudigt en voldoende expressief is om de gewenste voorbeelden uit te werken.  Recursieve functies zijn nodig om complexere functies te definiëren zoals bijvoorbeeld een implementatie van foldr:

\begin{equation} \label{eq: sumEx}
    \begin{split}
        \textbf{let\:rec}\:foldr\:= &\:l\:op\:mempty.\: \\
        & \textbf{do}\:n\leftarrow \textbf{empty}\:l; \\
        & \textbf{if}\:n\:\textbf{then}\:\textbf{return}\:mempty \\
        & \textbf{else} \\
        & \qquad \textbf{do}\:h \leftarrow \textbf{head}\:l;\\
        & \qquad \textbf{do}\:t \leftarrow \textbf{tail}\:l; \\
        & \qquad \textbf{do}\:y \leftarrow \textbf{foldr}\:t\:op\:mempty    ; \\
        & \qquad \textbf{return}\:(op\:h\:y)\:\textbf{in} \\
        & \qquad \qquad \textbf{foldr}\:(1:2:5:6:[\:\:])\:(+)\:[\:\:];\\
    \end{split}
\end{equation}

Dit voorbeeld programma telt de waarden in de lijst (1:2:5:6:[ ]) op tot 14.

\section{Conditie}
Een simpele conditie om de controlestroom van programma's te bepalen op basis van een test.

\begin{equation}
    \begin{split}
        & \textbf{if}\  true \  \textbf{then} \  \textbf{return} \  0 \  \textbf{else} \  \textbf{return} \  1 \\
        & \leadsto \textbf{return}\  0
    \end{split}
\end{equation}

\begin{equation}
    \begin{split}
        & \textbf{if}\  false \  \textbf{then} \  \textbf{return} \  0 \  \textbf{else} \  \textbf{return} \  1 \\
        & \leadsto \textbf{return}\  1
    \end{split}
\end{equation}





%%% Local Variables: 
%%% mode: latex
%%% TeX-master: "masterproef"
%%% End: 

\chapter{Operationele Semantiek}
\label{hoofdstuk:semantiek}
\begin{table}
    \centering
    \begin{tabular}{|l|}
        \hline
        % top
        \\
        % header
         \begin{tabular} {l r}
              \begin{tabular}{|l|}
              \hline
                     $c \leadsto c'$ \\
                \hline
              \end{tabular} & Computatie reductie \\
         \end{tabular} \\
         % rules
          \begin{tabular}{c}
            $\inference{}{(\lambda x.\:c)\:v \leadsto c\:[\:v\:/\:x\:]}[E-AppAbs] \qquad \inference{}{\textbf{let}\:x\: = \: v \: \textbf{in} \:c \leadsto c\:[\:v\:/\:x\:]}[E-Let]$ \\ 
            \\
            $\inference{}{\textbf{let\:rec}\:f\:=\:c_{1}\:\textbf{in}\:c_{2} \leadsto c_{2}\:[c_{1}\:[(\textbf{let\:rec}\:f\:=\:c_{1}\:\textbf{in}\:f)\:/\:f]\:/\:f]}[\hl{E-Letrec}]$ \\
            \\
            $\inference{c_{1} \leadsto c_{1}'}{\textbf{do}\:x \leftarrow c_{1}\:;\:c_{2} \leadsto \textbf{do}\:x \leadsto c_{1}'\:;\:c_{2}}[E-Do] \qquad \inference{}{\textbf{do}\:x \leftarrow \textbf{return}\:v\:;\:c_{2} \leadsto c_{2}\:[\:v\:/\:x\:]}[E-DoRet]$ \\
            \\
            $\inference{}{\textbf{do}\:x \leftarrow \textbf{op}\:l\:v\:(y.\:c_{1})\:;\:c_{2} \leadsto \textbf{op}\:l\:v\:(y.\: \textbf{do} \: x \leftarrow c_{1}\:;\:c_{2})}[E-DoOp]$ \\
            \\
            $\inference{}{\textbf{do}\:x \leftarrow \textbf{sc} \:l\:v\:(y.\:c_{1})\:(z.\:c_{2}) \: ;\: c_{3} \leadsto \textbf{sc} \:l\:v\:(y.\:c_{1})\:(z.\: \textbf{do} \: x \leftarrow c_{2} \: ;\: c_{3})}[E-DoSc]$ \\
            \\
            $\inference{}{\textbf{do}\:x \leftarrow \textbf{for}\:lstv\:(y.\:c_{1}) (z.\:c{2});\:c_{3} \leadsto \textbf{for}\:lstv\:(y.\:c_{1})\:(z.\:\textbf{do}\:x \leftarrow c_{2};\:c_{3})}[\hl{E-DoFor}]$ \\
            \\
            $\inference{c \leadsto c'}{h \star c \leadsto h \star c'}[E-Hand] \qquad \inference{(\textbf{return}\:x \mapsto c_{r}) \in h}{h \star \textbf{return} \: v \leadsto c_{r} \:[\:v\:/\:x\:]}[E-HandRet]$ \\
            \\
            $\inference{(\textbf{op}\:l\:x\:k \mapsto c) \in h}{h \star \textbf{op}\:l\:v\:(y.\:c_{1}) \leadsto c\:[\:v\:/\:x,\:(\lambda y.\:h \star c_{1}) \: / \: k]}[E-HandOp]$ \\
            \\
            $\inference{(\textbf{op}\:l\:\_\:\_) \notin h}{h \star \textbf{op}\:l\:v\:(y.\:c_{1}) \leadsto \textbf{op}\:l\:v\:(y.\: h \star c_{1})}[E-FwdOp]$ \\
            \\
            $\inference{(\textbf{sc}\:l\:x\:p\:k \mapsto c) \in h}{h \star \textbf{sc}\:l\:v\:(y.\:c_{1})\:(z.\:c_{2}) \leadsto c\:[\:v\:/\:x,\:(\lambda \: y. \: h \star\:c_{1}) \: / \: p, (\lambda z. \: h \star c_{2}) \:/\:k\:]}[E-HandSc]$ \\
            \\
            $\inference{(\textbf{sc}\:l\:\_\:\_\:\_) \notin h \\ (\textbf{fwd}\:f\:p\:k \mapsto c_{f}) \in h \qquad g\:=\:\lambda(p',\:k')\:.\:\textbf{sc}\:l\:v\:(y.\:p'\:y)\:(z.\:k'\:z)}{h \star \textbf{sc}\:l\:v\:(y.\:c_{1})\:(z.\:c_{2}) \leadsto c_{f}\:[\:(\lambda\:y.\:h \star c_{1})\:/\:p,\:(\lambda z.\: h \star c_{2})\:/\:k,\:g\:/\:f]}[E-FwdSc]$\\
            \\
            $\inference{(\textbf{for}\:lstv_{1}\:p\:k \mapsto c_{for}) \in h}{h \star \textbf{for}\:lstv_{2}\:(y.\:c_{1})\:(z.\:c_{2}) \leadsto c_{for}[lstv_{2}\:/\:lstv_{1},\:(y.\:h \star c_{1})\:/\:p, (z.\:h \star c_{2})\:/\:k]}[\hl{E-HandFor}]$\\
            \\
            $\inference{(\textbf{for}\:\_\:\_\:\_ \notin h)}{h \star \textbf{for}\:lstv\:(y.\:c_1)\:(z.\:c_2) \leadsto \textbf{for}\:lstv\:(y.\:h \star c_1)\:(z.\: h \star c_2)}[\hl{E-FwdFor}]$ 
          \end{tabular} \\
          % bottom
          \\
        \hline
    \end{tabular}
    \caption{Operationele semantiek van $\lambda_{sc}^{p}$}
    \label{fig:semantiek}
\end{table}

\begin{table}
    \centering
    \begin{tabular}{|l|}
        \hline
        \\
        \begin{tabular} {l r}
              \begin{tabular}{|l|}
              \hline
                     $c \leadsto c'$ \\
                \hline
              \end{tabular} & Computatie reductie \\
         \end{tabular} \\
         \begin{tabular}{c}
          $\inference{}{\textbf{map}\:f\:(v_{1}:v_{2}:\:...\::v_{n}:[\:\:]) \leadsto
          ((f\:v_{1}):(f\:v_{2}):\:...\::(f\:v_{n}):[\:\:])}[\hl{E-Map}]$\\
          \\
          $\inference{1 \leq i \leq n \qquad c_i \leadsto c_i'}{(c_1:...:c_i:...:c_{n}:[\:\:]) \leadsto (c_1:...:c_i':...:c_{n}:[\:\:])}[\hl{E-ParList}]$\\
          \\
          $\inference{}{((\textbf{return}\:v_1):(\textbf{return}\:v_2):...:(\textbf{return}\:v_n):[\:\:]) \leadsto \textbf{return}\:(v_1:v_2:...:v_n:[\:\:])}[\hl{E-ListRet}]$ \\
          \\
          $\inference{}{\textbf{head}\:(v_1:lstv) \leadsto \textbf{return}\:v_1}[\hl{E-HeadLstv}] \qquad \inference{}{\textbf{head}\:(c_1:lstc) \leadsto \textbf{return}\:c_1}[\hl{E-HeadLstc}]$ \\
          \\
        $\inference{}{\textbf{tail}\:(v_1:lstv) \leadsto \textbf{return}\:lstv}[\hl{E-TailLstv}] \qquad \inference{}{\textbf{tail}\:(c_1:lstc) \leadsto \textbf{return}\:lstc}[\hl{E-TailLstc}]$ \\
        \\
        $\inference{}{\textbf{empty}\:[\:\:] \leadsto \textbf{return}\:true}[\hl{E-EmptyTrue}]$ \\
        \\
        $\inference{}{\textbf{empty}\:(v:lstv) \leadsto \textbf{return}\:false}[\hl{E-EmptyLstvFalse}]$\\ 
        \\
        $\inference{}{\textbf{empty}\:(c:lstc) \leadsto \textbf{return}\:false}[\hl{E-EmptyLstcFalse}]$ \\
        \\
        $\inference{}{\textbf{fst}\:(v_1,\:v_2) \leadsto \textbf{return}\:v_1}[\hl{E-First}] \qquad \inference{}{\textbf{snd}\:(v_1,\:v_2) \leadsto \textbf{return}\:v_2}[\hl{E-Second}]$ \\
        \\
        $\inference{c_1 \leadsto c_1'}{\textbf{if}\:c_1\:\textbf{then}\:c_2\:\textbf{else}\:c_3 \leadsto \textbf{if}\:c_1'\:\textbf{then}\:c_2\:\textbf{else}\:c_3}[\hl{E-If}]$\\
        \\
        $\inference{}{\textbf{if}\:true\:\textbf{then}\:c_1\:\textbf{else}\:c_2 \leadsto c_1}[\hl{E-IfTrue}]$\\
        \\
        $\inference{}{\textbf{if}\:false\:\textbf{then}\:c_1\:\textbf{else}\:c_2 \leadsto c_2}[\hl{E-IfFalse}]$\\
        \\
         \end{tabular}
         \\
          \hline
    \end{tabular}
    \caption{Operationele semantiek van $\lambda_{sc}^{p}$, lijst- en hulp-functies}
    \label{tab:opSemLst}
\end{table}

Tabellen \ref{fig:semantiek} en \ref{tab:opSemLst} toont de operationele semantiek voor de $\lambda_{sc}^{p}$-calculus. De toegevoegde regels zijn gemarkeerd. De meest cruciale toevoegingen zijn de regels rond de toevoeging van het nieuwe sleutelwoord \textbf{for}, namelijk \textbf{E-DoFor} voor het aaneenrijgen van computaties, \textbf{E-HandFor} voor het behandelen van de \textbf{for}-constructie door de handler die dit behandelt en \textbf{E-FwdFor} voor het forwarden van \textbf{for}. Het inzicht dat dit niet generiek kan gebeuren, is hierbij belangrijk. Gelijkaardige regels waren nodig voor de \textbf{sc} en \textbf{op} sleutelwoorden met dezelfde functies. Verder zijn enkele toevoegingen gedaan om recursieve functie-behandeling mogelijk te maken (\textbf{E-Letrec}) en lijst- en andere hulp-functies (Tabel \ref{tab:opSemLst}) expliciet toe te voegen. De bestaande regels uit de $\lambda_sc$-calculus \cite{Bosman2022} blijven onveranderd.

\section{Letrec: Recursieve functies}
De \textbf{let rec} constructie biedt de mogelijk om recursieve functies te definiëren in de calculus. De clausule bestaat uit het \textbf{let rec} sleutelwoord gevolgd door een variabele-naam $f$ gevolgd door een computatie $c_{1}$ waar de variabele $f$ in kan voorkomen gevolgd door het sleutelwoord \textbf{in} gevolgd door de computatie waarin de variable $f$ kan voorkomen. Deze clausule wordt gereduceerd naar $c_{2}$ met f in $c_{2}$ vervangen door $c_{1}$ met $f$ vervangen door $\textbf{let\:rec}\:=\:c_{1}\:\textbf{in}\:f$. \textbf{E-Letrec} is een aangepaste versie van de regel beschreven in hoofdstuk twee van het boek "Principles of Programming Languages" \cite{Palmer2009}.

\section{For clausule}
Het \textbf{for} sleutelwoord laat toe om een computatie te mappen over een lijst van waarden. De gewenste semantiek om computaties met eventueel effecten te mappen over een lijst van waarden is niet mogelijk via een klassieke map functie omdat de computatie in dit geval niet puur is en alle relevante handlers niet noodzakelijk binnen de computatie zitten. Het gevolg is dat een klassieke map aanpak vastloopt op een normaalvorm (\textbf{return, op} of \textbf{sc}). Om een correcte semantiek voor het \textbf{for} sleutelwoord te bekomen zijn drie semantische regels nodig.

\subsection{E-DoFor}
\textbf{E-DoFor} laat toe om een \textbf{do} statement door te schuiven naar de continuatie van de \textbf{for} constructie en tegelijk de computatie na de \textbf{for} constructie binnen de continuatie te brengen. Deze regel is nodig om programma's correct aan elkaar te rijgen rond het gebruik van \textbf{for} constructies en is zeer gelijkaardig aan \textbf{E-DoRet}, \textbf{E-DoOp} en \textbf{E-DoSc}.

\subsection{E-HandFor}
\textbf{E-HandFor} laat toe om een \textbf{for} constructie te behandelen door een handler. De \textbf{for} constructie wordt hierbij vervangen door de $c_{for}$ computatie in de handler waarbij de handler eveneens naar binnen geschoven wordt in de te mappen functie en de continuatie. Elke handler kan de clausule implementeren of laten forwarden (\textbf{E-FwdFor}, \Cref{sec:fwdfor}). Aangezien deze behandeling nieuwe \textbf{for} constructies kan introduceren is het raadzaam om vooraan het programma een handler te implementeren die overgebleven pure \textbf{for} constructies afhandelt. De implementatie van deze clausule specifieert of de handler de constructie in parallel of sequentieel afhandelt.

\subsection{E-FwdFor} \label{sec:fwdfor}
\textbf{E-FwdFor} behandelt de forwarding voor het geval dat de \textbf{for} clausule niet door de handler geïmplementeerd wordt. De functie van de forwarding is tweevoudig. Deze regel schuift de handler binnen de computatie die te mappen is en de resumptie.

\section{Lijst- en hulp-functies}
Deze sectie behandelt toevoegingen aan de calculus die nodig zijn voor lijst-manipulatie en hulp-functies die de implementatie van de semantiek van de \textbf{for} clausule vergemakkelijken. 
\subsection{E-Map}
Via deze regel kan in de calculus een lijst van waarden worden omgevormd naar een lijst van computaties met behulp van een computatie $f$. Deze regel is essentieel om deze omvorming te maken. De \textbf{E-ListRet}-regel helpt bij de omvorming in de andere richting. 
\subsection{E-ParList}
\textbf{E-ParList} laat parallelle reductie van elementen in een lijst van computaties toe. De regel stelt dat een element in de lijst dat een stap kan maken, deze stap zet. Merk op dat dit een niet-deterministisch regel is. De voorwaarde om nog een stap te kunnen zetten is dat de computatie niet in normaalvorm is, met andere woorden niet in de vorm:
\begin{equation}
    \textbf{return}\:\_ \:\:|\:\:\textbf{op}\:\:\_\:\_\:\_\:\:|\:\:\textbf{sc}\:\_\:\_\:\_\:\_\:\:|\:\:\textbf{for}\:\_\:\_\:\_\:\_
\end{equation}

\subsection{E-ListRet}
Als een lijst van computaties gereduceerd kan worden naar een lijst van \textbf{return} clausules van waarden, dan kan de lijst vervangen worden door een \textbf{return} clausule van de lijst van waarden. Deze clausule laat, in combinatie met \textbf{E-DoRet}, toe om een lijst van computaties om te vormen naar een lijst van waarden.

\subsection{Lijst-manipulatie}
\textbf{E-HeadLstv}, \textbf{E-HeadLstc}, \textbf{E-TailLstv}, \textbf{E-TailLstc}, \textbf{E-EmptyTrue}, \textbf{E-EmptyLstvFalse} en \textbf{E-EmptyLstcFalse} zijn eenvoudige functies die lijst-manipulatie van lijsten van waarden en lijsten van computaties vergemakkelijken.

\subsection{Paar-manipulatie}
\textbf{E-First} en \textbf{E-Second} laten manipulatie van paren van waarden toe.

\subsection{If constructie}
\textbf{E-If}, \textbf{E-IfTrue} en \textbf{E-IfFalse} implementeren op standaardwijze een \textbf{if} clausule.


%%% Local Variables: 
%%% mode: latex
%%% TeX-master: "masterproef"
%%% End: 

\chapter{Type- en Effect-Systeem}
% TODO: for in computatie typering en handler typering
% TODO: wat met lijsten?
\label{hoofdstuk:typesysteem}
Dit hoofdstuk behandelt het type- en effect-systeem van de $\lambda^{p}_{sc}$-calculus. In bijzonder focust dit hoofdstuk zich op de toevoegingen aan het type-systeem van de $\lambda_{sc}$-calculus waar dit systeem op gebaseerd is. In het bijzonder behandelt dit typesysteem de syntaxis zoals beschreven in Figuur \ref{fig:syntaxisNodig}.

\section{Waarde typering}
Figuur \ref{fig:typeW} toont de typering voor waarden in de calculus. Deze figuur is identiek aan Figuur \ref{fig:syntaxisScoped} aangezien de nieuwe syntaxis geen waarden toevoegt.

\begin{table}
    \centering
    \begin{tabular}{|l|}
        \hline
        % top
        \\
        % header
         \begin{tabular} {l r}
              \begin{tabular}{|l|}
              \hline
                     $\Gamma \vdash v\::\:\sigma$ \\
                \hline
              \end{tabular} & Waarde typering \\
         \end{tabular} \\
         % rules
          \begin{tabular}{c}
          \\
            $\inference{(x\::\:\sigma) \in \Gamma}{\Gamma \vdash x\::\:\sigma}[T-Var] \qquad \inference{}{\Gamma \vdash (\:) \::\:(\:)}[T-Unit]$ \\ 
            \\
            $\inference{\Gamma \vdash v_1\::\:A \qquad \Gamma \vdash v_2 \::\:B}{\Gamma \vdash (v_1,\:v_2)\::\:(A,\:B)}[T-Pair]$\\
            \\
            $\inference{\Gamma,\:x\::\:A \vdash c\::\:\underline{C}}{\Gamma \vdash \lambda \: x.\:c\::\:A \rightarrow \underline{C}}[T-Abs] \qquad \inference{\Gamma \vdash v\::\:A \qquad A \equiv B}{\Gamma \vdash v\::\:B}[T-Eqv]$\\
            \\
            $\inference{\Gamma \vdash A \::\: * \qquad \Gamma \vdash v\::\: \forall \: \alpha.\:\sigma}{\Gamma \vdash v\::\:\sigma\:[\:A\:/\:\alpha\:]}[T-Inst] \qquad \inference{\Gamma,\:\alpha \vdash v\::\:\sigma \qquad \alpha \notin \Gamma}{\Gamma \vdash v\::\:\forall\:\mu.\:\sigma}[T-Gen]$\\
            \\
            $\inference{\Gamma \vdash v\::\:\forall\:\mu.\:\sigma}{\Gamma \vdash b \::\:\sigma\:[\:E\:/\:\mu\:]}[T-InstEff] \qquad \inference{\Gamma,\:\mu \vdash v\::\:\sigma \qquad \mu \notin \Gamma}{\Gamma \vdash v\::\:\forall \mu.\:\sigma}[T-GenEff]$\\
            \\
          \end{tabular} \\
          % bottom
          \\
        \hline
    \end{tabular}
    \caption{Waarde typering van $\lambda_{sc}^{p}$}
    \label{fig:typeW}
\end{table}

\section{Computatie typering}
Figuur \ref{fig:typeC} toont de typering voor computaties in de calculus. 

\subsubsection{Computatie typering}
\begin{table}
    \centering
    \begin{tabular}{|l|}
        \hline
        % top
        \\
        % header
         \begin{tabular} {l r}
              \begin{tabular}{|l|}
              \hline
                     $\Gamma \vdash c\::\:\underline{C}$ \\
                \hline
              \end{tabular} & Computatie typering \\
         \end{tabular} \\
         % rules
          \begin{tabular}{c}
          \\
            $\inference{\Gamma \vdash v_1 \::\:A \rightarrow \underline{C} \qquad \Gamma \vdash v_2 \::\:A}{\Gamma \vdash v_1\:v_2\::\:\underline{C}}[T-App]$ \\
            \\
            $\inference{\Gamma \vdash c_1\::\:A!\langle E \rangle \qquad \Gamma,\:x\::\:A \vdash c_2 \::\:B! \langle E \rangle}{\Gamma \vdash \textbf{do}\:x \leftarrow c_1;c_2\::\:B!\langle E \rangle}[T-Do]$\\
            \\
            $\inference{\Gamma \vdash c\::\:\underline{C} \qquad \underline{C} \equiv \underline{D}}{\Gamma \vdash c\::\:\underline{D}}[T-EqC]$ \\
            \\
            $\inference{\Gamma \vdash v\::\:\sigma \qquad \Gamma,\:x\::\:\sigma \vdash c\::\:\underline{C}}{\Gamma \vdash \textbf{let}\:x = v\:\textbf{in}\:c\::\:\underline{C}}[T-Let]$\\
            \\
            $\inference{\Gamma \vdash v\::\:A}{\Gamma \vdash \textbf{return}\:v\::\:A!\langle E \rangle}[T-Ret]$\\
            \\
            $\inference{\Gamma \vdash v\::\:\forall \:\alpha.\:\alpha!\langle E \rangle \Rightarrow M\:\alpha!\langle F \rangle \qquad \Gamma \vdash c\::\:A!\langle E \rangle}{\Gamma \vdash v \star c\::\:M\:A!\langle F \rangle}[T-Hand]$\\
            \\
            $\inference{(l\::\:A_l \rightarrowtriangle B_l) \in \Sigma \qquad \Gamma \vdash v\::\:A_l \qquad \Gamma,\:y\::\:B_l \vdash c\::\:A!\langle l;E \rangle}{\Gamma \vdash \textbf{op}\:l\:v\:(y.\:c)\::\:A!\langle l;E \rangle}[T-Op]$\\
            \\
            $\inference{(l\::\:A_l \rightarrowtriangle B_l) \in \Sigma \qquad \Gamma \vdash v\::\:A_l \\ \Gamma,\:y\::\:B_l\vdash c_1 \::\: B!\langle l;E \rangle \qquad \Gamma,\:z\::\:B \vdash c_2 \::\:A!\langle l;E \rangle}{\Gamma \vdash \textbf{sc}\:l\:v\:(y.\:c_1)\:(z.\:c_2)\::\:A!\langle l;E \rangle}[T-Sc]$\\
            \\
          \end{tabular} \\
          % bottom
          \\
        \hline
    \end{tabular}
    \caption{Computatie typering van $\lambda_{sc}$}
    \label{fig:typeC}
\end{table}

\subsection{T-For}

\section{Handler typering}
Figuur \ref{fig:typeH} toont de typering voor handlers in de calculus. 

\begin{table}
    \centering
    \begin{tabular}{|l|}
        \hline
        % top
        \\
        % header
         \begin{tabular} {l l l r}
              \begin{tabular}{|l|}
              \hline
                     $\Gamma \vdash oprs\::\:\underline{C}$ \\
                \hline
              \end{tabular} & \begin{tabular}{|l|}
              \hline
                     $\Gamma \vdash \textbf{fwd}\:f\:p\:k \mapsto c\::\:\underline{C}$ \\
                \hline
              \end{tabular} & \begin{tabular}{|l|}
              \hline
                     $\Gamma \vdash h\::\:\underline{C} \Rightarrow \underline{D}$ \\
                \hline
              \end{tabular} & Handler typering \\
         \end{tabular} \\
         % rules
          \begin{tabular}{c}
          \\
            $\inference{}{\Gamma \vdash .\::\:\underline{C}}[T-Empty]$ \\ 
            \\
            $\inference{\Gamma \vdash oprs \::\: \underline{C} \qquad (l\::\: A_l \rightarrowtriangle B_l) \in \Sigma \qquad \Gamma,\:x\::\:A_l,\:k\::\:B_l \rightarrow \underline{C} \vdash c \::\: \underline{C}}{\Gamma \vdash \textbf{op}\:l\:x\:k \mapsto c,\:oprs\::\: \underline{C}}[T-OprOp]$\\
            \\
            $\inference{\Gamma \vdash oprs\::\:M\:A!\langle E \rangle \qquad (l\::\:A_l \rightarrowtriangle b_l) \in  \Sigma) \\
            \Gamma, \: \beta ,\:x\::\:A_l , \:p\::\:B_l \rightarrow M\: \beta ! \langle E \rangle ,\:k\::\: \beta \rightarrow M\:A! \langle E \rangle \vdash c \::\: M\:A! \langle E \rangle}{\Gamma \vdash \textbf{sc}\:l\:v\:p\:k \mapsto c,\:oprs\::\:M\:A! \langle E \rangle}[T-OprSc]$\\
            \\
            $\inference{A_p = \alpha \rightarrow M\: \beta ! \langle E \rangle \qquad A_k = \beta \rightarrow M \: A!\langle E \rangle \qquad A_k' = M\: \beta \rightarrow M\:A!\langle E \rangle \\
            \Gamma,\: \alpha,\:\beta,\:p\::\:A_p,\:k\::\:A_k,\:f\::\:(A_p,\:A_k') \rightarrow M\:A!\langle E \rangle \vdash c_f \::\: M\:A!\langle E \rangle}{\Gamma \vdash \textbf{fwd}\:f\:p\:k \mapsto c_f \::\: M\:A!\langle E \rangle}[T-Fwd]$\\
            \\
            $\inference{\langle E \rangle = \langle labels \: (oprs);\:F \rangle \qquad \Gamma, \: \alpha \vdash \textbf{return} x \mapsto c_r \::\: M\:\alpha ! \langle F \rangle \\ \Gamma,\:\alpha \vdash oprs\::\: M \: \alpha !\langle F \rangle \qquad \Gamma,\: \alpha \vdash \textbf{fwd}\:f\:p\:k \mapsto c_f\::\:M\:\alpha ! \langle F \rangle}{\Gamma \vdash \textbf{handler}\:\{\:\textbf{return},\:oprs,\:\textbf{fwd}\:\}\::\:\forall\:\alpha.\:\alpha!\langle E \rangle \Rightarrow M\:\alpha ! \langle F \rangle}[T-Handler]$\\
            \\
          \end{tabular} \\
          % bottom
          \\
        \hline
    \end{tabular}
    \caption{Handler typering van $\lambda_{sc}$}
    \label{fig:typeH}
\end{table}

\subsection{T-OprFor}


%%% Local Variables: 
%%% mode: latex
%%% TeX-master: "masterproef"
%%% End: 

\chapter{Metatheorie}
\label{hoofdstuk:metatheorie}
... 

\section{Lemma's}
...

\section{Behoud}
...

\section{Vooruitgang}
...

%%% Local Variables: 
%%% mode: latex
%%% TeX-master: "masterproef"
%%% End: 

\chapter{Interpreter}
\label{hoofdstuk:interpreter}
...

%%% Local Variables: 
%%% mode: latex
%%% TeX-master: "masterproef"
%%% End: 

\chapter{Uitgewerkte Voorbeelden}
\label{hoofdstuk:voorbeelden}
% TODO: behoud van drie eigenschappen aantonen?
...

\section{Voorbeelden met Scoped Effecten}
...

\section{Voorbeelden met Scoped Effecten}
...

\section{Voorbeelden met Beide Effecten}
...

%%% Local Variables: 
%%% mode: latex
%%% TeX-master: "masterproef"
%%% End: 

\chapter{Evaluatie}
\label{hoofdstuk:evaluatie}
... 

\section{Bijdrage}
...

\section{Bruikbaarheid}
...

\section{Correctheid}
...

\section{Implementatie}
...

\section{Backwards compatibility}
...

\subsection{$\lambda_{sc}$}
...

\subsection{$\lambda_{p}$}
...

%%% Local Variables: 
%%% mode: latex
%%% TeX-master: "masterproef"
%%% End: 

\chapter{Gerelateerd Werk}
\label{hoofdstuk:gerelateerd}
... 

%%% Local Variables: 
%%% mode: latex
%%% TeX-master: "masterproef"
%%% End: 

\include{Tekst/Hoofdstukken/besluit.tex}

% Indien er bijlagen zijn:
\appendixpage*          % indien gewenst
\appendix
%TODO
%\chapter{De eerste bijlage}
\label{app:A}
In de bijlagen vindt men de data terug die nuttig kunnen zijn voor de
lezer, maar die niet essentieel zijn om het betoog in de normale tekst te
kunnen volgen. Voorbeelden hiervan zijn bronbestanden,
configuratie-informatie, langdradige wiskundige afleidingen, enz.

In een bijlage kunnen natuurlijk ook verdere onderverdelingen voorkomen,
evenals figuren en referenties\cite{h2g2}.

\section{Meer lorem}
Iets

\subsection{Lorem 15--17}
Iets

\subsection{Lorem 18--19}
Iets

\section{Lorem 51}
Iets

%%% Local Variables: 
%%% mode: latex
%%% TeX-master: "masterproef"
%%% End: 

% ... en zo verder tot
%\chapter{De laatste bijlage}
\label{app:n}
In de bijlagen vindt men de data terug die nuttig kunnen zijn voor de
lezer, maar die niet essentieel zijn om het betoog in de normale tekst te
kunnen volgen. Voorbeelden hiervan zijn bronbestanden,
configuratie-informatie, langdradige wiskundige afleidingen, enz.

\section{Lorem 20-24}
Iets

\section{Lorem 25-27}
Iets

%%% Local Variables: 
%%% mode: latex
%%% TeX-master: "masterproef"
%%% End: 


\backmatter
% Na de bijlagen plaatst men nog de bibliografie.
% Je kan de  standaard "abbrv" bibliografiestijl vervangen door een andere.
\bibliographystyle{abbrv}
\bibliography{referenties}

\end{document}

%%% Local Variables: 
%%% mode: latex
%%% TeX-master: t
%%% End: 
