\chapter{Type- en Effect-Systeem}
% TODO: weg For
% TODO: denk over T-For en T-OprFor
% TODO: misschien toch ook handige uitbreidingen typeren? booleans, letrec, condities, ...
% TODO: typering nalezen
\label{hoofdstuk:typesysteem}
Dit hoofdstuk behandelt het type- en effect-systeem van de $\lambda^{p}_{sc}$-calculus. De focus ligt hierbij op de \emph{
toevoegingen} aan het type-systeem van de $\lambda_{sc}$-calculus waar dit systeem op gebaseerd is. Het startpunt waarop dit systeem verderbouwt is beschreven in Sectie \ref{sec:typeScCalc}. In het bijzonder behandelt dit typesysteem de syntaxis zoals beschreven in Figuur \ref{fig:syntaxisNodig}. Aangezien het argument van deze thesis is dat de \emph{parallelle effecten een specifieke vorm van scoped effecten} zijn, zijn de toevoegingen beperkt tot de uitbreidingen voor de lijsten en de \textbf{map} constructie.

\section{Waarde typering}
\begin{table}
    \centering
    \begin{tabular}{|l|}
        \hline
        % top
        \\
        % header
         \begin{tabular} {l r}
              \begin{tabular}{|l|}
              \hline
                     $\Gamma \vdash v\::\:\sigma$ \\
                \hline
              \end{tabular} & Waarde typering \\
         \end{tabular} \\
         % rules
          \begin{tabular}{c}
          \\
            $\inference{(x : \sigma) \in \Gamma}{\Gamma \vdash x : \sigma}[T-Var] \qquad \inference{}{\Gamma \vdash ( \ ) : ( \ )}[T-Unit]$ \\ 
            \\
            $\inference{\Gamma \vdash v_1 : A \qquad \Gamma \vdash v_2 : B}{\Gamma \vdash (v_1 , \ v_2) : (A, \ B)}[T-Pair]$\\
            \\
            $\inference{\Gamma , \ x : A \vdash c : \underline{C}}{\Gamma \vdash \lambda \ x . \ c : A \rightarrow \underline{C}}[T-Abs] \qquad \inference{\Gamma \vdash v : A \qquad A \equiv B}{\Gamma \vdash v : B}[T-EqV]$\\
            \\
            $\inference{\Gamma \vdash v : \forall \ \alpha \ . \ \sigma \qquad \Gamma \vdash A }{\Gamma \vdash v : [ \ A \ / \ \alpha \ ] \ \sigma}[T-Inst] \qquad \inference{\Gamma , \ \alpha \vdash v : \sigma \qquad \alpha \notin \Gamma}{\Gamma \vdash v : \forall \ \alpha . \ \sigma}[T-Gen]$\\
            \\
            $\inference{\Gamma \vdash v : \forall \ \mu . \ \sigma \qquad \Gamma \vdash E}{\Gamma \vdash v : [ \ E \  / \ \mu \ ] \ \sigma}[T-InstEff] \qquad \inference{\Gamma , \ \mu \vdash v : \sigma \qquad \mu \notin \Gamma}{\Gamma \vdash v : \forall \ \mu . \ \sigma}[T-GenEff]$\\
            \\
            $\inference{}{\Gamma \vdash nilV[A] : ListV A}[\hl{T-NilV}] \qquad \inference{\Gamma \vdash v_1 : A \quad \Gamma \vdash c_2 : ListV A}{\Gamma \vdash consV[A] \  v_1 \  v_2 : ListV A}[\hl{T-ConsV}]$ \\
          \end{tabular} \\
          % bottom
          \\
        \hline
    \end{tabular}
    \caption{Waarde typering van $\lambda_{sc}^{p}$}
    \label{fig:typeW}
\end{table}
% Verwijzen naar verschil in syntaxis, enige toevoeging is lijst van waarden... lstv
% 2 nieuwe regels, voor lege lijst en voor lijst op te bouwen
Figuur \ref{fig:typeW} toont de typering voor waarden in de calculus. De toevoegingen tegenover Figuur \ref{fig:typeWaarde} zijn gemarkeerd. Merk op dat aangezien de calculus de syntaxis van waarden enkel uitbreidt met een lijst van waarden zoals aangeduid in Figuur \ref{fig:syntaxisScoped} de enige nieuwe type-regels betrekking hebben tot de typering van lijsten van waarden. De typering voor de lijsten onleent inspiratie van Sectie 11.12 uit TAPL\cite{Pierce2002}. \newline
\textbf{T-NilV} typeert de lege lijst van type A ($NilV[A]$) als een lijst van waarden met type A ($ListV \  A$). \newline
\textbf{T-ConsV} typeert de compositie van een waarde van type A met een bestaande lijst van waarden van type A tot een nieuwe lijst van waarden van type A.

\section{Computatie typering}

\begin{table}
    \centering
    \begin{tabular}{|l|}
        \hline
        % top
        \\
        % header
         \begin{tabular} {l r}
              \begin{tabular}{|l|}
              \hline
                     $\Gamma \vdash c\::\:\underline{C}$ \\
                \hline
              \end{tabular} & Computatie typering \\
         \end{tabular} \\
         % rules
          \begin{tabular}{c}
          \\
            $\inference{\Gamma \vdash v_1 : A \rightarrow \underline{C} \qquad \Gamma \vdash v_2 : A}{\Gamma \vdash v_1 \ v_2 : \underline{C}}[T-App]$ \\
            \\
            $\inference{\Gamma \vdash c_1 : A ! \langle E \rangle \qquad \Gamma , \ x : A \vdash c_2 : B ! \langle E \rangle}{\Gamma \vdash \textbf{do} \ x \leftarrow c_1; \ c_2 : B ! \langle E \rangle}[T-Do]$\\
            \\
            $\inference{\Gamma \vdash c : \underline{C} \qquad \underline{C} \equiv \underline{D}}{\Gamma \vdash c : \underline{D}}[T-EqC] \qquad \inference{\Gamma \vdash v : \sigma \qquad \Gamma , \ x : \sigma \vdash c : \underline{C}}{\Gamma \vdash \textbf{let} \ x = v \ \textbf{in} \ c : \underline{C}}[T-Let]$\\
            \\
            $\inference{\Gamma \vdash v : A \qquad \Gamma \vdash E}{\Gamma \vdash \textbf{return} \ v : A ! \langle E \rangle}[T-Ret] \qquad \inference{\Gamma \vdash v : \forall \ \alpha . \ \alpha ! \langle E \rangle \Rightarrow M \ \alpha ! \langle F \rangle \\ \Gamma \vdash c\::\:A!\langle E \rangle}{\Gamma \vdash v \star c\::\:M\:A!\langle F \rangle}[T-Hand]$\\
            \\
            $\inference{(l^{op} : A_{op} \rightarrowtriangle B_{op}) \in \Sigma \\ \Gamma \vdash v : A_{op} \qquad \Gamma , \ y : B_{op} \vdash c : A ! \langle E \rangle \qquad l^{op} \in E}{\Gamma \vdash \textbf{op} \ l^{op} \ v \ (y . \ c) : A ! \langle E \rangle}[T-Op]$\\
            \\
            $\inference{(l^{sc} : A_sc \rightarrowtriangle B_sc) \in \Sigma \\ \Gamma \vdash v : A_{sc} \qquad \Gamma , \ y : B_{sc} \vdash c_1 : B ! \langle E \rangle \qquad \Gamma , \ z : B \vdash c_2  : A ! \langle E \rangle \qquad l^{sc} \in E}{\Gamma \vdash \textbf{sc} \ l^{sc} \ v \ (y. \ c_1) \ (z. \ c_2) : A ! \langle E \rangle}[T-Sc]$\\
            \\ % TODO: z : B goed of is meer vrijheid nodig?
            %$\inference{(l^{for} : A_{for} \rightarrowtriangle B_{for}) \in \Sigma \\ \Gamma \vdash v : ListV \ A_{for} \qquad
            %\Gamma , \ y : B_{for} \vdash c_1 : B ! \langle E \rangle \qquad \Gamma , \ z : B \vdash c_2 : A!\langle E \rangle \qquad l^{for} \in E }{\Gamma \vdash \textbf{for} \ l^{for} \ v \ (y. \ c_1) \ (z. \ c_2) : A!\langle E \rangle}[\hl{T-For}]$\\
            %\\
            $\inference{}{\Gamma \vdash nilC[\underline{C}] : ListC \ \underline{C}}[\hl{T-NilC}] \qquad \inference{\Gamma \vdash c : \underline{C} \quad \Gamma \vdash lstC[\underline{C}] : \underline{C}}{\Gamma \vdash consC[\underline{C}] \ c \ lstC[\underline{C}] : ListC \ \underline{C}}[\hl{T-ConsC}]$\\
            \\
            $\inference{\Gamma \vdash c_1 : A \rightarrow C \qquad \Gamma \vdash v_2 : ListV \ A}{\textbf{map} \ v_1 \ v_2 : ListC \ \underline{C}}[\hl{T-Map}]$\\
            \\
          \end{tabular} \\
          % bottom
          \\
        \hline
    \end{tabular}
    \caption{Computatie typering van $\lambda_{sc}$}
    \label{fig:typeC}
\end{table}

Figuur \ref{fig:typeC} toont de typering voor computaties in de calculus overgenomen van de $\lambda_{sc}$-calculus met uitbreidingen voor lijsten van computaties. De volgende subsecties bespreken de toegevoegde regels.

%\subsection{T-For}
%Deze subsectie bespreekt de \textbf{T-For} regel die de parallelle effecten typeert. Analaag aan de \textbf{T-Op} en \textbf{T-Sc} regels levert het opzoeken van een het label $l^{for}$ in $\Sigma$ een signatuur $A_{for} \rightarrowtriangle B_{for}$ op. 

\subsection{Lijsten}
De regels \textbf{T-NilC} en \textbf{T-ConsC} die lijsten van computaties typeren zijn gelijkaardig aan de eerder besproken regels \textbf{T-NilV} en \textbf{T-ConsV} maar gegroepeerd onder de Figuur \ref{fig:typeC} aangezien lijsten van computaties zelf computaties zijn in de calculus.

%TODO: meer
\subsection{Map}
De regel \textbf{T-Map} typeert de \textbf{map} constructie die en is analoog aan een applicatie (\textbf{T-App}) maar over een lijst van waarden.

\section{Handler typering}
\begin{table}
    \centering
    \begin{tabular}{|l|}
        \hline
        % top
        \\
        % header
         \begin{tabular} {l l}
              \begin{tabular}{|l|}
              \hline
                     $\Gamma \vdash \textbf{return} \ x \mapsto c_r : M \ A ! \langle E \rangle$ \\
                \hline
              \end{tabular} & \begin{tabular}{|l|}
              \hline
                     $\Gamma \vdash oprs : M \ A ! \langle E \rangle$ \\
                \hline
              \end{tabular} \\ 
              \begin{tabular}{|l|}
              \hline
                     $\Gamma \vdash \textbf{fwd} \ f \ p \ k \mapsto c_f : M \ A ! \langle E \rangle$ \\
                \hline
              \end{tabular} & return-, operatie- en forwarding-typering \\
         \end{tabular} \\
         % rules
          \begin{tabular}{c}
          \\
            $\inference{\Gamma , \ x : A \mapsto c_r : M \ A ! \langle E \rangle}{\Gamma \vdash \textbf{return} \ x \mapsto c_r : M \ A ! \langle E \rangle}[T-Return] \qquad \inference{}{\Gamma \vdash .\::\:\underline{C}}[T-Empty]$\\
            \\
            $\inference{\Gamma \vdash oprs : M \ A ! \langle E \rangle \qquad (l^{op} : A_{op} \rightarrowtriangle B_{op}) \in \Sigma  \\ \Gamma , \ x : A_{op} , \ k : B_{op} \rightarrow M \ A ! \langle E \rangle \vdash c : M \ A ! \langle E \rangle}{\Gamma \vdash \textbf{op} \ l^{op} \ x \ k \mapsto c, \ oprs : M \ A ! \langle E \rangle}[T-OprOp]$\\
            \\
            $\inference{\Gamma \vdash oprs : M \ A ! \langle E \rangle \qquad (l^{sc} : A_{sc} \rightarrowtriangle B_{sc}) \in  \Sigma \\
            \Gamma, \ \beta , \ x : A_{sc} , \ p : B_{sc} \rightarrow M \ \beta ! \langle E \rangle , \ k : \beta \rightarrow M \ A ! \langle E \rangle \vdash c : M \ A ! \langle E \rangle}{\Gamma \vdash \textbf{sc} \ l^{sc} \ x \ p \ k \mapsto c, \ oprs : M \ A ! \langle E \rangle}[T-OprSc]$\\
            \\
            %$\inference{\Gamma \vdash oprs : M \ A ! \langle E \rangle \qquad (l^{for} : A_{for} \rightarrowtriangle B_{for}) \in \Sigma \\ \Gamma , \ \beta , \ x : ListV \ A_{for} , \ p : B_{sc} \rightarrow M \ \beta ! \langle E \rangle , \ k : \beta \rightarrow M \ A ! \langle E \rangle \vdash c : M \ A ! \langle E \rangle}{ \Gamma \vdash \textbf{for} \ l^{for} \ x \ p \ k \mapsto c, \ oprs : M \ A ! \langle E \rangle}[\hl{T-OprFor}]$\\
            %\\
            $\inference{A_p = \alpha \rightarrow M \ \beta ! \langle E \rangle \qquad A_p' = \alpha \rightarrow \gamma ! \langle E \rangle \\
            A_k = \beta \rightarrow M \ A ! \langle E \rangle \qquad A_k' = \gamma \rightarrow \delta ! \langle E \rangle \\
            \Gamma , \ \alpha , \ \beta , \ p : A_p , \ k : A_k , \ f : \forall \ \gamma \ \delta . \ (A_p, \ A_k') \rightarrow \delta ! \langle E \rangle \vdash c_f  : M \ A ! \langle E \rangle}{\Gamma \vdash \textbf{fwd} \ f \ p \ k \mapsto c_f : M \ A ! \langle E \rangle}[T-Fwd]$\\
            \\
            \end{tabular}
            \\
            \begin{tabular} {l l}
              \begin{tabular}{|l|}
              \hline
                     $\Gamma \vdash h : \forall \ \alpha . \ \alpha ! \langle E \rangle \Rightarrow M \ \alpha ! \langle F \rangle$ \\
                \hline
              \end{tabular} & handler-typering \\
         \end{tabular} \\
         \begin{tabular}{c}
         \\
            $\inference{\langle F \rangle \equiv_{\langle \rangle} \langle labels \ (oprs); \ E \rangle \qquad \Gamma , \ \alpha \vdash \textbf{return} \   x \mapsto c_r : M \ \alpha ! \langle E \rangle \\ \Gamma , \ \alpha \vdash oprs : M \ \alpha ! \langle E \rangle \qquad \Gamma , \ \alpha \vdash \textbf{fwd} \ f \ p \ k \mapsto c_f : M \ \alpha ! \langle E \rangle}{\Gamma \vdash \textbf{handler} \ \{ \textbf{return} \ x \mapsto c_r , \ oprs , \ \textbf{fwd} \ f \ p \ k \mapsto c_f \} : \forall \ \alpha . \ \alpha ! \langle F \rangle \Rightarrow M \ \alpha ! \langle E \rangle}[T-Handler]$\\
            \\
         \end{tabular}
         \\
          % bottom
          \\
        \hline
    \end{tabular}
    \caption{Handler typering van $\lambda_{sc}$}
    \label{fig:typeH}
\end{table}
Figuur \ref{fig:typeH} toont de typering voor handlers in de calculus. De typering is identiek zijn deze in $\lambda_{sc}$, geen veranderingen of toevoegingen zijn nodig aangezien het argument is dat mits toevoeging van de \textbf{map} constructie de $\lambda_{sc}$-calculus een equivalente semantiek bereikt als beschreven in $\lambda^p$ voor parallelle effecten of dat parallelle effecten een specialisatie vormen op scoped effecten.

% TODO: typering uitleggen ivm parallelle effecten


%%% Local Variables: 
%%% mode: latex
%%% TeX-master: "masterproef"
%%% End: 
